%%%%% Základní nastavení pro jednostranný tisk:
%%%%% Okraje: levý 40mm, pravý 25mm, horní a dolní 25mm (ale pozor, LaTeX si sám přidává 1in)
%%%%% ---------------------------------------------------------------
\documentclass[12pt, a4paper]{report}
\usepackage{geometry}
\setlength\textwidth{145mm}
\setlength\textheight{247mm}
\setlength\oddsidemargin{15mm}
\setlength\evensidemargin{15mm}
\setlength\topmargin{0mm}
\setlength\headsep{0mm}
\setlength\headheight{0mm}
\newcommand{\openright}{\clearpage}

%%%%% Základní nastavení pro oboustranný tisk:
%%%%% ---------------------------------------------------------------
% \documentclass[12pt, a4paper, twoside, openright]{report}
% \setlength\textwidth{145mm}
% \setlength\textheight{247mm}
% \setlength\oddsidemargin{15mm}
% \setlength\evensidemargin{0mm}
% \setlength\topmargin{0mm}
% \setlength\headsep{0mm}
% \setlength\headheight{0mm}
% \let\openright=\cleardoublepage

%%%%% Nastavení kódování vstupních souborů: UTF-8
%%%%% ---------------------------------------------------------------
\usepackage[utf8]{inputenc}

%%%%% Nastavení češtiny
%%%%% ---------------------------------------------------------------
\usepackage[czech]{babel}
\ifx\uv\undefined\newcommand{\uv}[1]{,,#1``}\fi

%%%%% Další užitečné balíčky
%%%%% ----------------------------------------------------------------
\usepackage{amsmath}                %%% rozšíření pro sazbu matematiky
\usepackage{amsfonts}               %%% matematické fonty
\usepackage{amsthm}                 %%% sazba vět, definic apod.
\usepackage{bm}                     %%% tučné symboly (příkaz \bm)
\usepackage{graphicx}               %%% vkládání obrázků
\usepackage{psfrag}                 %%% dodatečná úprava popisků v postscriptových obrázcích
\usepackage{fancyvrb}               %%% vylepšené prostředí pro strojové písmo
\usepackage{natbib}                 %%% zajištuje možnost odkazovat na reference stylem AUTOR (ROK), resp. AUTOR [ČÍSLO]
\usepackage{tikz}                   %%% vkládání vektorových obrázků
\usepackage{bbding}                 %%% balíček s nejrůznějšími symboly
\usepackage{icomma}                 %%% inteligetní čárka v matematickém módu
\usepackage{dcolumn}                %%% lepší zarovnání sloupců v tabulkách
\usepackage{booktabs}               %%% lepší vodorovné linky v tabulkách
\usepackage{paralist}               %%% lepší enumerate a itemize
%\usepackage{indentfirst}            %%% zaveď odsazení 1. odstavce
\usepackage[nottoc]{tocbibind}      %%% zajistí přidání seznamu literatury, obrázků a tabulek do obsahu
\usepackage[unicode]{hyperref}      %%% zajištuje generování hyperodkazů, bookmarků atp.
\usepackage{pdfpages}               %%% umožňuje vkládat PDF soubory do dokumentu
\usepackage{fancyhdr}               %%% umožňuje nastavit vlastní hlavičky a patičky stránky

%%%%% Nastavení hyperodkazů
%%%%% ------------------------------------------------------------
\hypersetup{pdftitle=Webové řešení na prodej vstupenek s~rezervací míst,
    pdfauthor=Filip Ditrich
    ps2pdf,
    colorlinks=false,               %% hyperlinky budou označeny červenými rámečky, které budou neviditelné při tisku na papír
    urlcolor=blue,
    pdfstartview=FitH,
    pdfpagemode=UseOutlines,
    pdfnewwindow,
    breaklinks                      %% zajistí, aby se dlouhé hyperodkazy mohly lámat přes více řádků
}

%%%%% Nastavení nadpisů
%%%%% ------------------------------------------------------------
\usepackage{titlesec}
%\titlespacing*{\chapter}{0pt}{-10mm}{5mm}
\titleformat{\chapter}{\normalfont\huge\bfseries}{\thechapter}{1em}{}

%%%%% Zkratky
%%%%% ---------------------------------------------------------------
\newcommand{\FIGDIR}{./figures}    %%% cesta do adresáře s obrázky

%%%%% Seznam použité literatury
%%%%% ---------------------------------------------------------------
%\bibliographystyle{czplainnat}    %% Autor (rok) s českými spojkami
%\bibliographystyle{plainnat}     %% Autor (rok) s anglickými spojkami
\bibliographystyle{unsrt}        %% [číslo]
\renewcommand{\bibname}{Seznam použité literatury}


%%%%% Použití fancyvrb (fancy verbatim) při definici prostředí pro
%%%%% sazbu kódu, resp. výstupů z počítačových programů
%%%%% ------------------------------------------------------------
\DefineVerbatimEnvironment{PCinout}{Verbatim}{fontsize=\small, frame=single}


%%%%% Hlavní část dokumentu
%%%%% ------------------------------------------------------------
\begin{document}
%%% titulní strana
    %%%
%%%  TITULNÍ STRANA
%%%
%%%  * soubor obsahující titulní stranu bakalářské práce
%%%
%%%  ===========================================================================
\pagestyle{empty}
\begin{center}

%%% název školy
{\bfseries\large UNICORN VYSOKÁ ŠKOLA s.r.o.}

    \vspace{5mm}

    %%% název oboru
    {\Large Softwarový vývoj}

    \vfill
    \vspace{5mm}

    %%% logo školy
    \centerline{\mbox{\includegraphics[width=83.3mm]{\FIGDIR/uu-icon}}}

    \vfill
    \vspace{5mm}

    %%% typ práce
    {\large\MakeUppercase{Bakalářská práce}}

    \vspace{15mm}

    %%% název práce
    {\LARGE\bfseries Webové řešení na prodej vstupenek s~rezervací míst}

    \vfill

    %%% autor a vedoucí práce
    \begin{tabular}{rl}
        Autor bakalářské práce: & Filip Ditrich\\
        \noalign{\vspace{2mm}}
        Vedoucí bakalářské práce: & Ing.\ Marek Beránek, Ph.D.\\
    \end{tabular}

    \vfill

    %%% rok
    Praha 2023

\end{center}

%%% TODO: kopie zadání
    \includepdf[pages={1}]{\FIGDIR/navrh-zadani.pdf}
    \includepdf[pages={2}]{\FIGDIR/navrh-zadani.pdf}
%%% čestné prohlášení
    %%%
%%%  ČESTNÉ PROHLÁŠENÍ
%%%
%%%  * soubor obsahující čestné prohlášení k bakalářské práci
%%%
%%%  ===========================================================================
\newpage
\pagestyle{empty}
\vspace*{\stretch{8}}

%%% nadpis
\noindent
{\large\bfseries Čestné prohlášení}\\

%%% text
\noindent
Prohlašuji, že jsem svou bakalářskou práci na~téma \textit{Webové řešení na prodej vstupenek s~rezervací míst}vypracoval samostatně pod~vedením vedoucího bakalářské práce a s~použitím výhradně odborné literatury a~dalších informačních zdrojů, které jsou v práci všechny citovány a~jsou také uvedeny v~seznamu použitých zdrojů.\\

\noindent
Jako autor této bakalářské práce dále prohlašuji, že v~souvislosti s~jejím vytvořením jsem neporušil autorská práva třetích osob a~jsem si plně vědom následků porušení ustanovení § 11 a následujících autorského zákona č.~121/2000~Sb.\\

\noindent
Dále prohlašuji, že odevzdaná tištěná verze bakalářské práce je shodná s~verzí, která byla odevzdána elektronicky.

%%% podpis - místo/den
\vspace{18mm}
\noindent
V \makebox[4cm]{\dotfill} dne \makebox[2.5cm]{\dotfill}
\hspace*{\fill}
\makebox[4cm]{\dotfill}

%%% podpis
\begin{flushright}
    \noindent
    Filip Ditrich
\end{flushright}

%%% poděkování
    %%%
%%%  PODĚKOVÁNÍ
%%%
%%%  * soubor obsahující poděkování za pomoc při vytvoření bakalářské práce
%%%
%%%  ===========================================================================
\newpage
\pagestyle{empty}
\vspace*{\stretch{8}}

%%% nadpis
\noindent
{\large\bfseries Poděkování}\\

%%% text
\noindent
Rád bych touto cestou poděkoval vedoucímu své bakalářské práce, panu Ing.~Markovi Beránkovi,~Ph.D., za věnovaný čas, spolupráci a konzultace při zpracování mé práce a zejména pak za cenné rady, podnětné připomínky a konstruktivní kritiku, kterou mi při vedení práce poskytl.

%%% první stránka
    %%%
%%%  PRVNÍ STRANA
%%%
%%%  * soubor obsahující první stranu bakalářské práce
%%%
%%%  ===========================================================================
\newpage
\pagestyle{plain}

%%% začátek stránkování
\setcounter{page}{6}

%%% nastavení speciálních okrajů
\newgeometry{textwidth=100mm, textheight=247mm, left=40.4mm, right=75mm}

%%% pozadí
\tikz[remember picture,overlay]
\node[opacity=1,inner sep=0pt] at (current page.center)
    {\includegraphics[width=\paperwidth,height=\paperheight]{\FIGDIR/side-banner}};

\begin{center}

    %%% logo školy
    \centerline{\mbox{\includegraphics[width=39.6mm]{\FIGDIR/uu-icon}}}

    \vfill

    %%% název práce v češtině
    \Large\textbf{Webové řešení na prodej vstupenek s~rezervací míst}

    \vspace{5mm}

    %%% název práce v angličtině
    \Large{Web-based ticketing solution with seat reservation}

    \vfill

    %%% logo školy
    \centerline{\mbox{\includegraphics[width=45mm]{\FIGDIR/uu-logo}}}

\end{center}
\restoregeometry

%%% abstrakt
    %%%
%%%  ABSTRAKT
%%%
%%%  * soubor obsahující abstrakt bakalářské práce
%%%
%%%  ===========================================================================
\newpage
\pagestyle{plain}

%%% abstrakt česky
\vbox to 0.5\vsize{
    \setlength\parindent{0mm}
    \setlength\parskip{5mm}

    %%% nadpis
    {\large\bfseries TODO: Abstrakt}

    %%% TODO: text abstrakt až po dopsání práce (lol)
    \noindent
    Tato bakalářská práce se zaměřuje na vývoj frontendové části webové aplikace pro prodej vstupenek s rezervací míst. Cílem práce je vyvinout prototyp aplikace, která potenciálnímu zákazníkovi umožní zobrazit mapu míst, vybrat si preferované místo, přidat vstupenky do nákupního košíku a následně vytvořit objednávku.

    Teoretická část práce se zaměřuje na obecnou problematiku prodeje vstupenek a moderního řešení pomocí platforem a služeb poskytujících online prodej vstupenek s možností rezervací míst. Tato část dále analyzuje trh současných poskytovatelů a definuje nejhlavnější technické problémy, které mohou při vývoji takového systému vzniknout a to výhradně z pohledu frontendu.

    Praktická část práce definuje rozsah implementované aplikace, podrobně popisuje hlavní funkce, komponenty, datové modely i některé nezbytné backendové části. Kapitola o návrhu uživatelského rozhraní popisuje principy, vzory a osvědčené postupy návrhu uživatelského rozhraní v kontextu vyvíjené aplikace. Kapitola o vývoji frontendové části poté podrobně popisuje technologie, nástroje a knihovny použité při vývoji aplikace.


    \vss}

%%% abstrakt anglicky
\nobreak\vbox to 0.49\vsize{
    \setlength\parindent{0mm}
    \setlength\parskip{5mm}

    %%% nadpis
    {\large\bfseries TODO: Abstract}

    %%% TODO: text anglicky
    \noindent
    This bachelor thesis deals with the implementation of a ticketing solution using seat reservations. The aim of
    the theoretical part of the thesis is to elaborate this issue and identify the main technical challenges and requirements for the implementation
    of such a system, especially from the frontend perspective. These challenges, such as intuitive user interface, availability of information
    real-time sales information, administration and management of seating plans or final booking and payment of orders, will be discussed in this thesis
    will be described in this thesis and options for their solution will be presented. The practical part will focus on the implementation of the part of the web application dealing with
    rendering of the interactive seating plan and its interaction with the customer. In this part the user interface design will be presented
    together with an introduction of the selected technologies to be implemented. Emphasis will be placed on the optimized design of the seating plan, the specification of
    of its data format, communcation with the API and overall documentation of the components used in the resulting application. The aim of this work is to describe
    the issues and challenges in implementing a web application addressing seat ticketing and present part of a real solution for such a
    application especially from the perspective of an interactive seat selection plan.

    Keywords: interactive seating plan, seat reservations, tickets, web applications, JavaScript, React, SVG
    \vss}


%%% obsah
    \newpage
    \pagestyle{empty}
    \tableofcontents
%%% TODO: jednotlivé kapitoly
    %%%%% Úvod
%%%%% ------------------------------------------------------------
\chapter*{Úvod}
\addcontentsline{toc}{chapter}{Úvod}

%%% Sekce - Prodej vstupenek
%%%%% Wording: ✅
%%%%% Styling: ✅
%%%%% References: ✅
%%%%% Grammar: ✅
%%% --------------------------------------------------------------
\section*{Prodej vstupenek}
\addcontentsline{toc}{section}{Prodej vstupenek}
\label{sec:uvod-prodej-vstupenek}
Prodej vstupenek na kulturní a jiné různé události je důležitou součástí zábavního průmyslu, neboť poskytuje lidem přístup na koncerty, divadelní představení, sportovní či jiné události.
Prodej vstupenek umožňuje pořadatelům těchto akcí nejen kontrolovaný průběh akce, ale především generuje dostatečný finanční tok peněz před konáním jejich akce.
Tyto finance zpravidla potřebují pro zajištění všech potřebných prostředků pro uspořádání akce a pro pořadatele se tedy jedná o jeden z klíčových faktorů úspěchu konání akce.
Potřebují tedy pro zákazníky zajistit co nejsnadnější a nejpříjemnější možnost nákupu vstupenek.

S nástupem moderních technologií se online prodej vstupenek proměnil v atraktivní a preferovaný způsob jejich nákupu, jelikož zákazníkům umožňuje snadný, pohodlný, a hlavně rychlý způsob platby, aniž by se museli kamkoliv fyzicky dostavit.
Tento nový moderní způsob prodeje vstupenek však nabízí výhody nejen zákazníkům, ale také pořadatelům akcí.
Systémy, které jsou na tomto způsobu prodeje vstupenek založeny, pořadatelům akcí umožňují bezproblémový prodej vstupenek, což vede k efektivnějšímu plánování a řízení akcí.
Tyto systémy pořadatelům také nabízejí cenné údaje o zákaznících, jejich preferencích a chování, které mohou využít při plánování marketingových strategií, cílených reklam či propagačních akcí za účelem zvýšení zapojení zákazníků a podpoření prodeje.

Jedním z nejvýznamnějších pokroků v této oblasti online řešení prodeje vstupenek bylo rozšíření o možnost rezervace míst v prostoru konání akce.
Toto řešení nově zákazníkům umožňuje zarezervovat si místo na dané události, což opět přináší několik výhod pro zákazníky, ale také pro pořadatele akcí.
Zákazníkům umožňuje rezervaci a výběr místa, které je pro ně nejvhodnější.
Pořadatelům akcí pak rezervace míst umožňuje předem plánovat kapacitu dané akce a také zjistit, jaké místo je pro zákazníky nejvíce preferované.
Dále také značně snižuje počet možných podvodů se vstupenkami, jelikož kapacita je jasně dána počtem míst k sezení a nelze ji snadno překročit.

Webová řešení prodeje vstupenek s rezervací míst se v posledních letech stávají stále více oblíbenými a využívanými v různých odvětvích, včetně zábavního průmyslu, sportu či cestování.
Avšak s rapidním vývojem v oblasti webových technologií je důležité sledovat a využívat nové trendy a technologie a přizpůsobovat jim takováto řešení, aby byla pro zákazníky stále atraktivní a relevantní.
Tato práce se proto zaměřuje na vývoj frontendové části webové aplikace prodeje vstupenek s rezervací míst, která bude využívat moderní webové technologie a nástroje, které jsou v současné době nejvíce využívané a oblíbené.

%%% Sekce - Cíle práce
%%%%% Wording: ✅
%%%%% Styling: ✅
%%%%% References: ✅
%%%%% Grammar: ✅
%%% --------------------------------------------------------------
\section*{Cíle práce}
\addcontentsline{toc}{section}{Cíle práce}
\label{sec:uvod-cile-prace}
Cílem této práce je vyvinout prototyp responzivní webové aplikace nabízející prodej vstupenek s rezervací míst se zaměřením převážně na vývoj frontendové části.

Výsledkem této práce bude webová aplikace vyvinuta moderními webovými nástroji a technologiemi, která umožní potenciálním zákazníkům zobrazit mapu areálu nějaké akce či kulturní události, vybrat si jedno či více preferovaných míst, přidat si vstupenky do nákupního košíku a vytvořit tak objednávku.
Tato práce se bude zabývat vývojem takového webového řešení, ale pouze z pohledu frontendové části.
Ostatní funkčnosti, jako například backednový systém či administrační řešení, nebudou součástí této práce.

Nejprve bude ale pro vývoj třeba prozkoumat a analyzovat existující řešení prodeje vstupenek s rezervací míst, které jsou v současné době využívány.
Na základě identifikace klíčových částí takovýchto systémů budou následně vytvořeny dílčí uživatelské příběhy aplikace, které budou sloužit jako základ pro návrh uživatelského rozhraní.

Aby byla práce považována za úspěšně dokončenou, musí být splněny následující cíle:

\begin{itemize}
    \item Byly identifikovány klíčové prvky a funkčnosti webových řešení prodeje vstupenek s rezervací míst.
    \item Byl vytvořen návrh uživatelského rozhraní aplikace na základě definovaných uživatelských příběhů.
    \item Byla vyvinuta responzivní webová aplikace prodeje vstupenek s rezervací míst.
    \item Byla aplikace nasazena do produkčního prostředí.
\end{itemize}

    %%%%% Kapitola 1 - Analýza existujících řešení
%%%%% ------------------------------------------------------------
\chapter{Analýza existujících řešení}
\label{chap:analyza-trhu}
TODO: průzkum poskytovatelů, seznam jejich funkčností, strategií, taktik, popis, výhody a nevýhody

%%% Sekce - Ticketmaster
%%% --------------------------------------------------------------
\section{Ticketmaster}
\label{sec:analyza-trhu-ticketmaster}
TODO: popis Ticketmasteru

%%% Sekce - Ticketportal
%%% --------------------------------------------------------------
\section{Ticketportal}
\label{sec:analyza-trhu-ticketportal}
TODO: popis Ticketportalu

%%% Sekce - GoOut
%%% --------------------------------------------------------------
\section{GoOut}
\label{sec:analyza-trhu-goout}
TODO: popis GoOutu

%%% Sekce - NFCtron
%%% --------------------------------------------------------------
\section{NFCtron}
\label{sec:analyza-trhu-nfctron}
TODO: popis NFCtronu

    %%%%% Kapitola 3 - Specifikace prototypu
%%%%% ------------------------------------------------------------
\chapter{Specifikace prototypu}
\label{chap:specifikace}

Praktická část pojednává o~vývoji prototypu frontendu webové aplikace pro prodej vstupenek s~důrazem na implementaci funkčnosti rezervace míst. Nutno podotknout že výsledný prototyp nebude a ani není v plánu, aby byl plně funkční, nýbrž pouze ukazuje možnou implementaci konrkétních zvolených částí.\\

K implementaci prototypu je důležité předem vydefinovat jasnou specifikaci a požadavky na výsledný produkt. Bez těchto speicifkací by nebylo možné finální výsledek objektivně zhodnotit. Po jasné specifiaci požadavků bude potřeba prototyp vizuálně navrhnout a připravit jako podklad k implementaci. K té bude také třeba zanalyzovat konkrétní požadavky a zvolit správné technologie. Tento prototyp bude realizován pouze z frontendové části, tedy z pohledu vizuálního rozhraní pro potenciálního zákazníka, který si bude chtít zakoupit vstupenku s využitím rezervace místa. Z tohoto důvodu bude alespoň minimálně popsána funkčnost dostupného backednového rozhraní, který bude sloužit jako zdroj dat pro frontend.\\

Všechny zmíněné postupy budou v této části práce blíže popsány a vysvětleny v jednotlivých kapitolách.

%%% Sekce - Hlavní funkčnosti
%%% --------------------------------------------------------------
\section{Hlavní funkčnosti}
\label{sec:hlavni-funkcnosti}
TODO: Obecný popis funkčností

%%% Sekce - Rozhraní mapy
%%% --------------------------------------------------------------
\section{Rozhraní mapy}
\label{sec:sepcifikace-mapa}
TODO: Popis rozhraní mapy, sektorů, sedaček, míst ke stání, atd.

%%% Sekce - Vstupenky
%%% --------------------------------------------------------------
\section{Vstupenky}
\label{sec:specifikace-vstupenky}
TODO: Popis vstupenek, jejich typů, model, atributy, atd.

%%% Sekce - Datový model mapy
%%% --------------------------------------------------------------
\section{Datový model mapy}
\label{sec:specifikace-datovy-model}
TODO: Popis datového modelu mapy, formátu, atd.

%%% Sekce - Backend
%%% --------------------------------------------------------------
\section{Backend}
\label{sec:specifikace-backend}
TODO: Popis backendu, jeho funkcí a poskytovaných API endpointů

    %%%%% Kapitola 4 - Návrh uživatelského rozhraní
%%%%% ------------------------------------------------------------
\chapter{Návrh uživatelského rozhraní}
\label{ch:navrh-uzivatelskeho-rozhrani}
Ve světě digitálních produktů a jejich designu jsou uživatelské rozhraní, z anglického \foreign{\acf{ui}}, a uživatelský zážitek, z anglického \foreign{\acf{ux}}, dva pojmy, které se často zaměňují, ačkoli se jedná o velmi odlišné aspekty procesu vývoje produktu.
Tato kapitola si klade za cíl představit koncepty \ac{ui} a \ac{ux}, prozkoumat jejich vzájemný vztah a zabývat se specifiky návrhu \ac{ui} pro aplikaci pro prodej vstupenek s rezervací míst.

\textbf{\ac{ui}} se vztahuje k vizuálním prvkům produktu, se kterými uživatel interaguje – tedy tlačítkům, textu, ikonografii, formulářům a všem vizuálním prvkům, které umožňují uživateli interagovat s produktem.
V kontextu aplikace pro prodej vstupenek s rezervací míst se \ac{ui} vztahuje například k interaktivnímu plánu sedaček, výběru vstupenek, tlačítku pro přechod k dokončení objednávky nebo nákupnímu košíku.

\textbf{\ac{ux}} je na druhou stranu celkový zážitek uživatele při interakci s produktem.
Je ovlivněn snadností použití, hodnotou, kterou uživatel z produktu získává, a emocemi, které jsou při interakci vyvolány.
\ac{ux} bere v potaz celou cestu uživatele, od okamžiku, kdy uživatel do aplikace vstoupí, až po okamžik, kdy dokončí nákup.

Významným aspektem \ac{ux} je uživatelská cesta, z anglického \foreign{User Journey}, která popisuje cestu uživatele při interakci s produktem.
Uživatelská cesta se skládá z jednotlivých kroků, které uživatel musí absolvovat, aby dosáhl svého cíle.

Souhra \ac{ui} a \ac{ux} je v procesu návrhu produktu klíčová.
Dobře navržené \ac{ui} usnadňuje \ac{ux}.
Například intuitivně navržený plán sedadel (\ac{ui}) může proces výběru sedadla zpříjemnit a zjednodušit (\ac{ux}).

Následující sekce této kapitoly prozkoumají základní principy návrhu \ac{ui} a možná použití Maslowovy hierarchie, za účelem návrhu rozhraní více zaměřeného na uživatele.
Dále budou uvedeny a porovnány různé nástroje, které jsou k dispozici pro návrh \ac{ui} a důvody rozhodnutí pro konkrétní nástroj.
Následně budou analyzovány specifikace prototypu z kapitoly~\ref{ch:specifikace} z hlediska \ac{ui}/\ac{ux} se zaměřením na tzv.\ uživatelské příběhy, které tvoří základ \ac{ux} designu.

Závěr této kapitoly bude věnován návrhu interaktivního plánu sedaček, který je klíčovým \ac{ui} prvkem vyvýjeného prototypu aplikace.

%%% Sekce - Principy návrhu uživatelského rozhraní
%%% --------------------------------------------------------------
\section{Principy návrhu uživatelkého rozhraní}
\label{sec:navrh-principy}
Návrh uživatelského rozhraní je poměrně rozsáhlá disciplína, která se zaměřuje na vizuální a interaktivní aspekty produktu.
Při návrhu \ac{ui} je důležité dodržovat určité principy, které zajišťují optimální uživatelskou zkušenost.
Tato sekce shrnuje některé základní principy návrhu \ac{ui} a posuzuje jejich implikace v kontextu aplikace pro prodej vstupenek s rezervací míst.

\textbf{Konzistence}: Tento princip prosazuje zachování jednotnosti napříč všemi prvky \ac{ui}.
Konzistence se projevuje v použití podobných prvků, akcí a designu napříč celým rozhraním.
Například pokud určitá barva značí interaktivní prvek na plánu sedadel, stejná barva by měla být použita i pro značení interaktivních prvků jinde v rámci aplikace.
Tímto se zvyšuje předvídatelnost, což uživatelům usnadňuje orientaci a navigaci v rozhraní.

\textbf{Uživatel v kontrole}: Základním principem návrhu \ac{ui} je umožnit uživateli cítit se vždy v kontrole nad produktem.
Toho lze dosáhnout návrhem transparentního a intuitivního systému, ve kterém uživatel vždy ví, kde se nachází a jak postupovat.
V kontextu aplikace pro prodej vstupenek to může znamenat poskytnutí jasného a zřejmého způsobu, jak uživatelé mohou přejít k výběru sedadla, přidání do košíku a dokončení objednávky.

\textbf{Zpětná vazba}: Zpětná vazba je klíčovým aspektem každé interakce, protože potvrzuje nebo informuje  uživatele o vykonaných akcích.
Vizuální indikátory, jako je zvýraznění vybraného sedadla nebo potvrzovací zpráva při přidání vstupenky do košíku, poskytují uživateli okamžitou zpětnou vazbu.
Tím se snižuje nejistota a zvyšuje se důvěra uživatele v rozhraní.

\textbf{Jednoduchost}: Návrh \ac{ui} by měl směřovat k jednoduchosti.
Čím méně úsilí musí uživatel vynaložit na pochopení rozhraní, tím lepší bude celková uživatelská zkušenost.
Čisté, jednoduché rozhraní s jasným zaměřením na funkčnost snižuje kognitivní zátěž a zvyšuje použitelnost.

\textbf{Prevence a řešení chyb}: Chyby jsou nevyhnutelné v jakékoli interakci, ale dobře navržené \ac{ui} může zabránit většině uživatelských chyb nebo zjednodušit jejich řešení.
To může znamenat například zakázání tlačítka \textit{Pokračovat} dokud není vybráno sedadlo nebo zobrazení jasných a užitečných chybových zpráv, když něco selže.

\textbf{Afordance a signifikance}: \foreign{Afordance} se vztahuje k vlastnosti objektu, která naznačuje, jak se má používat.
\foreign{Signifikance} jsou vizuálními nápovědami k témto \foreign{afordancím}.
Například sedadlo na plánu sedadel může být navrženo tak, aby naznačovalo, že na něj lze kliknout (\foreign{afordance}), a změna kurzoru při najetí na sedadlo (\foreign{signifikance}) může tuto zprávu posílit.

Pochopení a aplikace těchto základních principů návrhu \ac{ui} je klíčové pro vytvoření intuitivního a uživatelsky přívětivého rozhraní.
Tyto principy řídí rozhodnutí v rámci návrhu a pomáhají návrhu \ac{ui} s celkovým cílem poskytnout uživatelům bezproblémový zážitek z rezervace vstupenek.
Další sekce se zabývá tím, jak lze hierarchii Maslowa aplikovat pro další zlepšení uživatelsky orientovaného návrhu.

%%% Sekce - Aplikovaná psychologie na návrh uživatelských rozhraní
%%% --------------------------------------------------------------
\section{Aplikovaná psychologie na UI/UX}
\label{sec:navrh-psychologie}

\epigraph{\textit{``Some people say, "Give the customers what they want." But that's not my approach. Our job is to figure out what they're going to want before they do.''}}{-- Steve Jobs}

Proces návrhu uživatelského rozhraní se netýká pouze estetiky nebo funkcionality v izolaci.
Ve skutečnosti, k vytvoření rozhraní, které skutečně rezonuje s uživateli, si lze vypůjčit koncept z psychologie - Maslowovu hierarchii potřeb.
Tato hierarchie, obvykle vizualizovaná jako pyramidová struktura, ilustruje cestu jednotlivce k seberealizaci a naplnění, začínající od základních fyziologických potřeb až po složitější emoční a psychologické potřeby.

\textit{Maslowova hierarchie potřeb} je teorie psychologa Abrahama Maslowa, která se snaží vysvětlit, co motivuje lidi.
Maslow tvrdil, že lidé mají potřeby, které se snaží uspokojit, ale některé z nich jsou naléhavější než jiné.
Když jsou tyto potřeby uspokojeny, lidé se mohou cítit šťastnější, ale když nejsou, lidé mohou být frustrovaní a nespokojení.\cite{maslow}
Maslow rozdělil lidské potřeby do pěti základních úrovní, které jsou znázorněny na obrázku~\ref{fig:maslow} níže.

\begin{figure}[H]
    \centering
    \includegraphics[width=0.8\textwidth]{\FIGDIR/maslow}
    \caption{Maslowova hierarchie potřeb\cite{wiki_potreby}}
    \label{fig:maslow}
\end{figure}

\begin{enumerate}
    \item \textbf{Fyziologické potřeby}: základní potřeby pro přežití, jako je potrava, voda, teplo a spánek
    \item \textbf{Potřeby bezpečí}: potřeby, které se týkají bezpečnosti a zabezpečení
    \item \textbf{Sociální potřeby}: potřeby, které se týkají příslušnosti, lásky a přátelství
    \item \textbf{Potřeby uznání}: potřeby, které se týkají úcty a sebeúcty
    \item \textbf{Potřeby seberealizace}: potřeby, které se týkají osobního růstu a rozvoje
\end{enumerate}

Jak to tedy ale souvisí s návrhem \ac{ui} a zejména s návrhem aplikace pro prodej vstupenek?

\textbf{Maslowova hierarchie potřeb} může být aplikována na návrh \ac{ui} tak, že každá úroveň hierarchie představuje jeden základní aspekt návrhu \ac{ui}.

V roce 2010 navrhl Steven Bradley v článku \textit{Designing For A Hierarchy Of Needs} podobnou hierarchii specificky pro design, se pěti odpovídajícími úrovněmi znázorněnými na obrázku~\ref{fig:design-hierarchy-of-needs}.\cite{bradley_hierarchy_of_needs}

\begin{figure}[H]
    \centering
    \includegraphics[width=0.8\textwidth]{\FIGDIR/design-hierarchy-of-needs}
    \caption{Hierarchie potřeb v návrhu \ac{ui} dle Stevena Bradleyho\cite{bradley_hierarchy_of_needs}}
    \label{fig:design-hierarchy-of-needs}
\end{figure}

\textbf{Funkčnost}: Na základě pyramidy jsou základní fyziologické potřeby.
V kontextu návrhu \ac{ui} to znamená základní funkčnost.
Aplikace musí fungovat tak, jak se očekává, aby si uživatelé mohli vybrat sedadlo, přidat vstupenku do košíku a dokončit proces objednávky bez jakýchkoli problémů.
Základní funkčnost musí být spolehlivá a robustní.

\textbf{Spolehlivost}: Další úroveň pyramidy je bezpečnost, která se v návrhu \ac{ui} týká spolehlivosti.
Rozhraní by mělo být navrženo tak, aby se uživatelé cítili bezpečně a sebevědomě při interakci s ním.
Poskytování jasných pokynů, okamžité zpětné vazby a potvrzení o úspěšných akcích (například přidání vstupenky do košíku) přispívá k pocitu bezpečí a použitelnosti.

\textbf{Použitelnost}: Střední část pyramidy pokrývá sociální potřeby, které se v \ac{ui} termínech rovnají uživatelské spokojenosti.
Esteticky příjemné rozhraní, personalizovaný uživatelský zážitek a interaktivní prvky (jako interaktivní plán sedadel) mohou významně zvýšit uživatelskou spokojenost.

\textbf{Odbornost}: Potřeby sebeúcty zahrnují touhu po uznání a respektu.
V kontextu aplikace pro prodej vstupenek by to mohlo znamenat přidání funkcí, které překračují očekávání uživatelů a zpříjemňují jim zážitek.
Může se jednat o něco tak jednoduchého, jako je blahopřání po úspěšném nákupu, nebo vizuální animace při výběru sedadla.

\textbf{Kreativita}: Na vrcholu pyramidy se nachází seberealizace, která se týká realizace osobního potenciálu a hledání osobního růstu a vrcholných zážitků.
Uživatelské rozhraní by mohlo přispět k této potřebě tím, že uživatelům umožní kreativně řešit problémy a dosáhnout svých cílů.
Například nabízení návrhů na nejlepší dostupná sedadla nebo podobných akcí může uživatele posílit a zlepšit jejich zážitek.

Použití Maslowovy hierarchie pro návrh \ac{ui} aplikace pro prodej vstupenek může pomoci zajistit, aby návrh splňoval potřeby uživatelů na různých úrovních.
Z počátku je nutné zajistit základní funkčnosti a spolehlivost, aby uživatelé mohli využívat aplikaci bez jakýchkoli problémů.
Dále je nutné zaměřit se na použitelnost, aby byl proces výběru sedadla a nákupu vstupenky co nejvíce zjednodušen.
Při postupu v hierarchii se budou zkoumat různé metody, jak zvýšit uživatelskou spokojenost a zlepšit jejich zážitek.
Cílem na vrcholu tohoto procesu je navrhnout rozhraní, které vyvažuje praktičnost a uživatelskou přívětivost, zatímco zároveň zajišťuje vizuální přitažlivost a emoční zapojení.
To povede k přínosnějšímu, uspokojivějšímu a úspěšnějšímu uživatelskému zážitku.

Další sekce se bude zabývat nástroji dostupnými pro návrh \ac{ui} a o důvodech pro výběr konkrétního nástroje pro tento projekt.

%%% TODO: Sekce - Nástroje pro návrh
%%% --------------------------------------------------------------
\section{Nástroje pro návrh}
\label{sec:navrh-ui-nastroje}
V oblasti návrhu uživatelského rozhraní má návrhář k dispozici širokou škálu nástrojů.
Tyto nástroje usnadňují nízkoúrovňové i vysokoúrovňové prototypování, přičemž každý z nich představuje jedinečnou sadu vlastností přispívajících k tvorbě, spolupráci a testování návrhů.
Tato sekce stručně popisuje tři nejčastěji používané nástroje pro návrh uživatelského rozhraní, a to Figma, Adobe XD a Sketch.

%%% TODO: Podsekce - Figma
%%% --------------------------------------------------------------
\subsection{Figma}
\label{subsec:navrh-ui-nastroje-figma}
Figma je nástroj pro návrh uživatelského rozhraní, který funguje v prohlížeči a je založen na cloudových technologiích.
Jeho hlavními výhodami jsou platformní nezávislost a snadná spolupráce.
Figma je také vybavena sadou funkcí, které usnadňují návrh uživatelského rozhraní, jako je vektorové kreslení, prototypování a předávání vývojářům.
\cite{figma}

\begin{figure}[H]
    \centering
    \includegraphics[width=0.8\textwidth]{\FIGDIR/figma}
    \caption{Ukázka nástroje Figma\cite{figma}}
    \label{fig:figma}
\end{figure}

%%% TODO: Podsekce - Adobe XD
%%% --------------------------------------------------------------
\subsection{Adobe XD}
\label{subsec:navrh-ui-nastroje-adobe-xd}
Adobe XD je nástroj od společnosti Adobe pro návrh uživatelského rozhraní, který funguje na platformách Windows i MacOS.
Jeho hlavními výhodami jsou jednoduché uživatelské rozhraní, prototypování a snadná integrace s ostatními produkty Adobe Suite.
\cite{adobe-xd}

\begin{figure}[H]
    \centering
    \includegraphics[width=0.8\textwidth]{\FIGDIR/adobe-xd}
    \caption{Ukázka nástroje Adobe XD\cite{adobe-xd}}
    \label{fig:adobe-xd}
\end{figure}

%%% TODO: Podsekce - Sketch
%%% --------------------------------------------------------------
\subsection{Sketch}
\label{subsec:navrh-ui-nastroje-sketch}
Sketch je nástroj pro návrh uživatelského rozhraní, který funguje výhradně na platformě MacOS.
Je to vektorový nástroj, který je chválen pro svou jednoduchost a rychlost.
Je užitečný při tvorbě rozhraní, webových stránek a ikon, i když absence vestavěných prototypovacích schopností může být pro některé návrháře omezujícím faktorem.
\cite{sketch}

\begin{figure}[H]
    \centering
    \includegraphics[width=0.8\textwidth]{\FIGDIR/sketch}
    \caption{Ukázka nástroje Sketch\cite{sketch}}
    \label{fig:sketch}
\end{figure}

%%% TODO: Podsekce - Výběr nástroje
%%% --------------------------------------------------------------
\subsection{Výběr nástroje}
\label{subsec:navrh-ui-nastroje-vyber}
Po podrobném zhodnocení byl pro návrh uživatelského rozhraní vyvíjené aplikace na prodej vstupenek s rezervací míst vybrán nástroj \textbf{Figma} z několika důvodů:

\textbf{Cloudově založený}: Figma umožňuje snadný přístup k návrhu z jakéhokoli zařízení, čímž odpadá nutnost instalace jakéhokoliv softwaru.
Nezáleží tedy ani na operačním systému, postačí pouze webový prohlížeč a připojení k internetu.

\textbf{Spolupráce v reálném čase}: I když se jedná o samostatný projekt, funkce spolupráce v reálném čase se ukazuje jako výhodná při poptávání zpětné vazby od možných budoucích zákazníků nebo konzultanta, čímž se zefektivňuje proces návrhu.

\textbf{Prototypování}: Rozsáhlé prototypovací schopnosti Figma usnadňují tvorbu interaktivních prototypů s vysokou kvalitou a důvěryhodností.

\textbf{Bezplatný}: Figma nabízí bezplatný plán, který je dostatečný pro většinu projektů.
Jedná se tedy o velmi výhodné řešení pro projekt, který není komerční či pro začínající návrháře.

\textbf{Vlastní zkušenost}: Osobně jsem měl možnost pracovat se všemi třemi nástroji a Figma se ukázala jako nejvhodnější pro tento projekt.
Zejména z důvodu jednoduchého uživatelského rozhraní, lehkosti použití a rychlosti prototypování.

%%% Sekce - Analýza specifikací a návrh UI
%%% --------------------------------------------------------------
\section{Analýza specifikací a návrh UI}
\label{sec:navrh-ui-analyza}
TODO: analýza specifikací, návrh UI, rozpad do user story, komponenty, interakce

%%% Sekce - Návrh UI mapy
%%% --------------------------------------------------------------
\section{Návrh UI mapy}
\label{sec:navrh-ui-mapa}
TODO: navrh rozhraní mapy, sedadel, hlavní komponenta

    %%%%% Kapitola 5 - Implementace frontendu
%%%%% ------------------------------------------------------------
\chapter{Implementace frontendu}
\label{chap:implementace-frontendu}

%%% Sekce - Analýza specifikací
%%% --------------------------------------------------------------
\section{Analýza specifikací}
\label{sec:implementace-analyza}
TODO: analýza specifikací z technického hlediska, popis problémů a možnosti jejich řešení

%%% Sekce - Výběr technologií
%%% --------------------------------------------------------------
\section{Výběr technologií}
\label{sec:implementace-vyber-technologii}
TODO: výběr technologií, popis výhod a nevýhod, základní popis

%%% Sekce - Vytvoření projektu
%%% --------------------------------------------------------------
\section{Vytvoření projektu}
\label{sec:implementace-vytvoreni-projektu}
TODO: popis postupu vytvoření projektu, co bylo použito, jak, atd.

%%% Sekce - Základní struktura projektu
%%% --------------------------------------------------------------
\subsection{Základní struktura projektu}
\label{sec:implementace-vytvoreni-projektu-zakladni-struktura}
TODO: popis základní struktury projektu, co obsahuje, jak je strukturován složkami a soubory atd.

%%% Sekce - Pomocná knihovna
%%% --------------------------------------------------------------
\subsection{Pomocná knihovna}
\label{sec:implementace-vytvoreni-projektu-pomocna-knihovna}
TODO: popis vlastní obecné pomocné knihovny, co obsahuje (utils, hooks, atd.)

%%% Sekce - Implementace nákupního košíku
%%% --------------------------------------------------------------
\section{Implementace nákupního košíku}
\label{sec:implementace-kosik}
TODO: popis implementace nákupního košíku, jeho data, správa, funkčnosti, atd.\\
TODO: další podkapitoly

%%% Sekce - Implementace mapy
%%% --------------------------------------------------------------
\section{Implementace mapy}
\label{sec:implementace-mapa}
TODO: popis implementace mapy, jaké jsou její části, jaké jsou její vlastnosti, atd.\\
TODO: další podkapitoly

%%% Sekce - Dokuemntace
%%% --------------------------------------------------------------
\section{Dokumentace}
\label{sec:implementace-dokumentace}
TODO: technická dokumentace jednotlivých komponent, jak spolu fungují, jaké jsou jejich vlastnosti, atd.

    % TODO: dopsat zaver uf
%%% Kapitola – Výzvy a problémy
%%%%%%% Wording: ⏳
%%%%%%% Styling: ⏳
%%%%%%% References: ⏳
%%% --------------------------------------------------------------
\chapter*{Závěr}
\addcontentsline{toc}{chapter}{Závěr}
\label{ch:zaver}
TODO: zhodnocení finálního prototypu oproti specifikacím, popis dalších možností vývoje, závěr

%%% seznam použité literatury
    \bibliography{main}
%%% seznam obrázků
    \listoffigures
%%% seznam tabulek
    \listoftables
%%% zkratky
    %%% TODO
%%% Použité zkratky v bakalářské práci, včetně jejich vysvětlení.
%%% --------------------------------------------------------------
\chapter*{Seznam použitých zkratek}
\addcontentsline{toc}{chapter}{Seznam použitých zkratek}
\begin{description}
    \item[API] Application Programming Interface
    \item[SVG] Scalable Vector Graphics
    \item[HTML] HyperText Markup Language
\end{description}

%%% přílohy
    %%% TODO
%%% Přílohy k bakalářské práci, existují-li (různé dodatky jako výpisy programů,
%%% diagramy apod.). Každá příloha musí být alespoň jednou odkazována z vlastního
%%% textu práce. Přílohy se číslují.
\chapter*{Přílohy}
\addcontentsline{toc}{chapter}{Přílohy}

\end{document}
