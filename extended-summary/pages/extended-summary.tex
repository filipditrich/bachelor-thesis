%%% Chapter - Extended Summary
%%%%% Wording: ⏳
%%%%% Styling: ⏳
%%%%% References: ⏳
%%%%% Grammar: ⏳
%%% --------------------------------------------------------------
\chapter{Extended Summary}
\label{ch:extended-summary}
The rapid advancement of technology has had a significant impact on various industries, including the entertainment sector, where ticket sales play a crucial role.
The emergence of online ticketing has not only transformed customer behavior by offering a quick, convenient, and user-friendly method of purchasing tickets, but also provided organizers with enhanced efficiency in event planning and management.
A noteworthy development in online ticketing is the ability to reserve specific seats at venues, which offers customers a personalized experience and provides organizers with detailed planning information and effective capacity management.
Additionally, this feature helps reduce the occurrence of ticket fraud, making it highly desirable.
Web-based ticketing with seat reservations has been adopted in numerous sectors, such as entertainment, sports, and travel.
However, keeping such solutions relevant to customers in the face of rapid advancements in web technology poses a challenge.
This thesis focuses on utilizing the latest technologies to create a frontend solution for a web-based ticketing and seat reservation application.

%%% Section - Goals and objectives
%%%%% Wording: ⏳
%%%%% Styling: ⏳
%%%%% References: ⏳
%%%%% Grammar: ⏳
%%% --------------------------------------------------------------
\begin{section}{Goals and objectives}
    \label{sec:goals-and-objectives}
    The main purpose of this work was to create a prototype for a responsive web-based application that enables ticket sales through seat reservations, with a main focus on front-end development.
    Built using modern web tools, the app allows potential customers to explore the event map, select their preferred seat, add tickets to cart and complete their order.
    While the focus was on the front end, understanding and analysis of existing ticketing solutions was critical to the development process.
    Therefore, the first step is to identify the key components of such a system and then create user stories that guide the design of the user interface.
    The goals that define the success of this project are:

    \begin{itemize}
        \item Identification of essential elements and functionalities of a web-based ticketing solution with seat reservations.
        \item Creation of the application's user interface design based on defined user stories.
        \item Development of a responsive web application enabling ticket sales with seat reservations.
        \item Deployment of the application in a production environment.
    \end{itemize}
\end{section}

%%% Section - Employed methods
%%%%% Wording: ⏳
%%%%% Styling: ⏳
%%%%% References: ⏳
%%%%% Grammar: ⏳
%%% --------------------------------------------------------------
\begin{section}{Employed methods}
    \label{sec:employed-methods}
    This thesis employed a combination of research, comparative analysis, and practical development.
    In the initial stages, a comprehensive examination of prevalent ticketing solutions with seat reservation capabilities was conducted.
    This examination entailed identifying the essential elements, functionalities, and prevailing patterns of these solutions.

    The design phase utilized a comparative analysis of design tools, specifically Adobe XD, Figma, and Sketch.
    The decision for their selection was determined by considering their features, user-friendliness, and alignment with project specifications.

    In the course of the implementation phase, an extensive examination of documentation was undertaken for the selected web tools and libraries, including React.js, TypeScript, Next.js, Mantine UI, and Tailwind CSS.
    These technologies were chosen based on their resilience, appropriateness for the project, and support from the community.

    In summary, this thesis relied on a combination of research, comparative analysis, and practical development as its primary methodologies.
\end{section}

%%% Section - Major sources
%%%%% Wording: ⏳
%%%%% Styling: ⏳
%%%%% References: ⏳
%%%%% Grammar: ⏳
%%% --------------------------------------------------------------
\begin{section}{Major sources}
    \label{sec:major-sources}
    This thesis has been informed and enriched by a multitude of resources.
    Some have provided deep dives into technical subjects, while others have given guidance on design principles and psychological theory.
    Here are the most important sources and how they have contributed to this work:

    MDN (Mozilla Developer Network): As a comprehensive resource for web technologies such as HTML, CSS, and JavaScript, MDN has served as the backbone for the technical aspects of this thesis.
    The insights from MDN on API interfaces and technologies such as WebSocket have provided the necessary grounding for building web applications.

    Maslow's Hierarchy of Needs: Maslow's theory of human motivation has been instrumental in understanding user behaviors and needs in the context of web development and design.
    This psychological framework has guided the user-centric approach adopted in this thesis.

    CSS Tricks: This site has been a valuable resource for mastering front-end design and CSS.
    It has provided practical tips and solutions for achieving the desired aesthetic and functionality in the developed web applications.

    Designing for a Hierarchy of Needs by Steven Bradley: This book has seamlessly bridged the gap between Maslow's psychological theory and design practice.
    It has provided a framework to ensure that the design of the web applications developed during this thesis meets users' needs at various levels.

    Figma, Adobe XD, and Sketch Comparison: This comparison has been fundamental in understanding and choosing the right design tools for the project.
    The insights gleaned from this comparison influenced the selection of the most suitable tool for prototyping and design, enhancing the design process.

    React and Next.js Documentation: These sources have been pivotal in understanding and implementing React and Next.js, the core technologies used in the web application development for this thesis.
    These technologies have been chosen for their scalability, ease of use, and robust community support, and the documentation has provided a reliable guide throughout the development process.

    Dribbble's UI Design Principles: These principles have been a guide in ensuring the development of user-friendly and intuitive interfaces.
    The principles have informed the design process by underlining the importance of elements like consistency, predictability, and aesthetics.

    Each of these resources has played a critical role in shaping the thesis, either by providing technical knowledge, design principles, project management guidelines, or understanding user behaviors and needs.
\end{section}

%%% Section - Summary
%%%%% Wording: ⏳
%%%%% Styling: ⏳
%%%%% References: ⏳
%%%%% Grammar: ⏳
%%% --------------------------------------------------------------
\begin{section}{Summary}
    \label{sec:summary}
    It has also been shown that well-thought-out design, careful planning and the right use of technology can lead to the creation of an application that exceeds the user's expectations The first part of the work focused on the identification of key parts of such systems with special emphasis on their user interface.
    An important topic was the understanding of user stories, their effective creation and transformation into real elements of the user interface.

    The interactive map, shopping cart and order completion show good responsiveness, on different devices, which clearly contributes to the overall user experience.
    While the frontend offers a smooth user experience, the absence of a comprehensive backend system leaves room for improvement and expansion in the future.
    The system is designed to adapt to different screen sizes, allowing access to the application from different devices without losing functionality.

    Each improvement would help to make the application more robust, versatile and user-friendly, bringing the application to the level of a professional ticketing system.
    This expansion would not only mean scaling up the existing system, but would require strategic planning to create a more sophisticated navigation structure.
    This structure would ensure seamless switching between different locations and events while maintaining an intuitive interface that is central to the user experience.

    At the start of the project, four objectives were established.

    The main goal was to determine the important components and features of the system used for selling tickets with reserved seats.
    These components were identified and explained in chapter~\fullref{ch:identification}, specifically the interactive seat map, the shopping cart, and the process of completing an order.

    The second objective was to create a user interface based on user stories.
    These user stories were collected in a specific chapter and were used to design the interface in another section.
    The outcome is the design of the user interface using the Figma tool, which is included in the appendix.

    The third objective was to create a web application for selling tickets with reserved seats.
    This objective was completed in the implementation chapter, which provides a detailed explanation of the application's development and structure.
    The source code for the application is included in the appendix.

%    FIXME
    The fourth goal of deploying the application in the production environment was successfully accomplished in chapter~\ref{ch:implementation}, which detailed the process of deploying the application on the Vercel platform in section~\ref{sec:implementation-deployment}.

    The goals have been achieved and now there is a fully functional web-based ticketing and seating application that can be accessed at https://seating-map.vercel.app.

    Overall, this work has been an impressive exploration of various aspects of web application development, including user interface design and front-end development techniques.
    It involved creating a comprehensive ticketing system with a user-friendly interface, which helped apply theoretical concepts to practical solutions.
    The importance of user-oriented design and the role of new technologies in simplifying complex tasks were emphasized.
    Furthermore, the work highlighted the significance of proper planning and organization in managing large-scale development tasks.
    It demonstrated that project architecture and the right choice of technologies are crucial factors that can greatly impact development speed, efficiency, and the overall quality of the final product.
    In conclusion, this work summarizes the process of creating a user-centered web application and emphasizes the value of user experience in front-end development and user interface design.
    The app has provided a strong foundation for future ticketing and seat reservation solutions, with potential for growth and improvement to create more intuitive and efficient apps.
\end{section}
