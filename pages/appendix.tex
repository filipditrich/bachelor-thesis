%%% Přílohy k bakalářské práci, existují-li (různé dodatky jako výpisy programů,
%%% diagramy apod.). Každá příloha musí být alespoň jednou odkazována z vlastního
%%% textu práce. Přílohy se číslují.
\appendix
\addtocontents{toc}{\protect\setlength{\cftsecnumwidth}{22mm}}
\chapter*{Seznam příloh}
\addcontentsline{toc}{chapter}{Seznam příloh}
\renewcommand{\thesection}{Příloha \Alph{section}}

%%% Příloha - Návrh uživatelského rozhraní
%%%%% File preparation: ⏳
%%%%% Instructions: ✅
%%% --------------------------------------------------------------
\begin{section}{Návrh uživatelského rozhraní}
    \label{appendix:ui-design}
    Návrh uživatelského rozhraní v~prostředí Figma je k nalezení v~přiloženém CD v~souboru \texttt{seating-map-figma.zip}.
    Archiv obsahuje soubor \texttt{seating-map.fig} a~jeho jednotlivé obrazovky exportované ve formátu PDF ve složce \texttt{/export}.
\end{section}
\newpage

%%% Příloha - Zdrojový kód aplikace
%%%%% File preparation: ⏳
%%%%% Instructions: ✅
%%% --------------------------------------------------------------
\begin{section}{Zdrojový kód aplikace}
    \label{appendix:source-code}
    Zdrojový kód aplikace je k~nalezení v~přiloženém CD v~archivu \texttt{seating-map-source.zip}.
    Pro spuštění aplikace je nutné mít nainstalované prostředí Node.js\footnote{\url{https://nodejs.org/en}} a nástroj pnpm\footnote{\url{https://pnpm.io/installation}}.

    Postup pro spuštění aplikace:

    \begin{enumerate}
        \item Rozbalte archiv \texttt{seating-map-source.zip}.
        \subitem \texttt{unzip seating-map-source.zip}
        \item Otevřete složku \texttt{seating-map-source} v~terminálu.
        \subitem \texttt{cd seating-map-source}
        \item Spusťte aplikaci příkazem \texttt{pnpm demo}\footnote{Příkaz nainstaluje potřebné balíčky, vytvoří potřebný \texttt{.env} soubor, sestaví aplikaci a sputí ji lokálně na portu \texttt{3008}}.
        \subitem \texttt{pnpm demo}
        \item Aplikace bude dostupná na adrese \url{http://localhost:3008}
    \end{enumerate}

    Aplikaci je také možné navštívit online na adrese \url{https://seating-map.vercel.app}.
\end{section}
