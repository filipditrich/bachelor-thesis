%%% Sekce – Výběr technologií
%%%%% Wording: ⏳
%%%%% Styling: ⏳
%%%%% References: ⏳
%%% --------------------------------------------------------------
\section{Výběr technologií}
\label{sec:implementace-techologie}
V této kapitole je zaměření na technologický stack vybraný pro frontendovou implementaci online platformy pro prodej vstupenek.
Tento výběr není náhodný; každá technologie byla vybrána z určitého důvodu, ať už je to kvůli konkrétní funkčnosti, kterou poskytuje, její synergie s ostatními technologiemi ve stacku, nebo kvůli její robustnosti a spolehlivosti.
Následující sekce podrobně popisují každou z těchto technologií.

%%% Podsekce – React.js
%%%%% Wording: ⏳
%%%%% Styling: ⏳
%%%%% References: ⏳
%%% --------------------------------------------------------------
\subsection{React.js}
\label{subsec:implementace-techologie-react}
% TODO: open-source footnote explanation
React.js je open-source JavaScriptová knihovna, vyvíjená a udržovaná společností Facebook, která usnadňuje tvorbu uživatelských rozhraní (UI) pro jednostránkové aplikace tím, že umožňuje vývojářům vytvářet znovupoužitelné komponenty uživatelského rozhraní.
Výběr React.js pro tento projekt je ovlivněn několika faktory.

Zaprvé, knihovna se vyznačuje svou funkcí virtuálního DOM, která zvyšuje rychlost a efektivitu vykreslování složitých uživatelských rozhraní.
Virtuální DOM Reactu, na rozdíl od tradičního modelu plného obnovení, umožňuje knihovně minimalizovat přímé manipulace s DOM, což šetří cenný výpočetní čas a zajišťuje plynulejší a výkonnější rozhraní.

Zadruhé, React.js je založen na komponentách, což znamená, že složité uživatelské rozhraní lze rozdělit na menší, znovupoužitelné komponenty.
Tento přístup zvyšuje spravovatelnost kódu a zjednodušuje ladění a testování.
To je zvláště užitečné v kontextu platformy pro prodej vstupenek, kde by několik prvků uživatelského rozhraní mohlo být znovu použito v různých částech aplikace.

Finálně, React.js má silnou podporu komunity, s množstvím zdrojů a nástrojů k dispozici.
Tato podpora je obrovským přínosem při vývoji složitých aplikací a může výrazně snížit čas potřebný k řešení potenciálních problémů.

%%% Podsekce – TypeScript
%%%%% Wording: ⏳
%%%%% Styling: ⏳
%%%%% References: ⏳
%%% --------------------------------------------------------------
\subsection{TypeScript}
\label{subsec:implementace-techologie-typescript}
TypeScript je staticky typovaná nadmnožina JavaScriptu, která do jazyka přidává volitelné typy.
TypeScript, vyvinutý a udržovaný společností Microsoft, nejenže zvyšuje schopnosti JavaScriptu, ale také pomáhá při tvorbě aplikací velkého rozsahu s větší spolehlivostí a menší náchylností k chybám.

Zatímco JavaScript je mocný jazyk, často je kritizován pro svou uvolněnost, zejména pokud jde o jeho dynamický typový systém.
TypeScript se snaží tento problém vyřešit tím, že představuje statické typování, což snižuje chyby za běhu a umožňuje lepší vývojové nástroje, jako je automatické dokončování, odvozování typů a typová kontrola.

Jazyk TypeScript se v posledních letech rychle rozšířil a stává se standardem pro vývoj velkých JavaScriptových aplikací.
Pro tento projekt je preferován kvůli jeho větší spolehlivosti a protože dobře spolupracuje s React.js.

%%% Podsekce – Next.js
%%%%% Wording: ⏳
%%%%% Styling: ⏳
%%%%% References: ⏳
%%% --------------------------------------------------------------
\subsection{Next.js}
\label{subsec:implementace-techologie-nextjs}
Next.js je open-source React framework, který zjednodušuje proces nastavení a zahrnuje několik užitečných funkcí, jako je server-side rendering a static site generation.
Framework je navržen tak, aby zvýšil výkon a efektivitu React aplikací a vybavil vývojáře nástroji pro tvorbu aplikací s bohatými funkcemi.

V kontextu platformy pro prodej vstupenek byl Next.js vybrán pro své funkce, které zjednodušují běžné úkoly.
Například Next.js má vestavěnou podporu pro API routes, které umožňují vývojářům vytvářet backend API ve stejném repozitáři jako frontend, což je ideální pro simulaci backendové funkcionality.

Next.js dále poskytuje funkce jako routování souborového systému, automatické rozdělení kódu a předvykreslování z krabice, což snižuje množství potřebné konfigurace a nastavení a umožňuje vývojářům soustředit se na logiku aplikace.

%%% Podsekce – Mantine UI
%%%%% Wording: ⏳
%%%%% Styling: ⏳
%%%%% References: ⏳
%%% --------------------------------------------------------------
\subsection{Mantine UI}
\label{subsec:implementace-techologie-mantine}
Mantine je komplexní knihovna komponent pro React s důrazem na použitelnost, přístupnost a zkušenosti vývojářů.
Zahrnuje řadu komponent, které lze kombinovat a přizpůsobit k vytvoření složitých a krásných uživatelských rozhraní.
Tato knihovna nabízí jednodušší způsob tvorby uživatelských rozhraní ve srovnání s vytvářením každého komponentu od základů, což může být časově náročné a náchylné k chybám.

Použitím knihovny komponent, jako je Mantine, se zajistí, že komponenty použité v celé aplikaci jsou konzistentní a snižuje se potřeba opakovaného kódu.
Dále se díky důrazu Mantine na přístupnost zajišťuje, že aplikace je použitelná co nejširšímu publiku.

Díky vynikající dokumentaci a bohatému výběru komponentů je Mantine preferovanou volbou pro projekt před jinými knihovnami, jako je MaterialUI\@.

%%% Podsekce – Tailwind CSS
%%%%% Wording: ⏳
%%%%% Styling: ⏳
%%%%% References: ⏳
%%% --------------------------------------------------------------
\subsection{Tailwind CSS}
\label{subsec:implementace-techologie-tailwind}
Tailwind CSS je CSS framework založený na utility-first přístupu pro rychlé vytváření vlastních designů.
Na rozdíl od tradičních CSS frameworků, které poskytují hotové komponenty, umožňuje Tailwind vývojářům vytvářet vlastní designy poskytováním nízkoúrovňových utility tříd, které mohou kombinovat k vytváření jedinečných rozhraní.

V projektu platformy pro prodej vstupenek je Tailwind CSS použit především pro rozložení a rychlé prototypování.
Jeho utility-first přístup zjednodušuje proces tvorby vlastních stylů a jeho responzivní modifikátory umožňují vytvářet designy, které fungují na všech velikostech obrazovky.

%%% Podsekce – Ostatní technologie
%%%%% Wording: ⏳
%%%%% Styling: ⏳
%%%%% References: ⏳
%%% --------------------------------------------------------------
\subsection{Ostatní technologie}
\label{subsec:implementace-techologie-ostatni}
Kromě hlavního technologického stacku diskutovaného výše bylo při tvorbě platformy pro prodej vstupenek klíčových několik dalších nástrojů a knihoven.
Každý z nich slouží k jedinečnému účelu a zlepšuje vývojový proces v oblastech jako je kvalita kódu, získávání dat, animace, správa formulářů a pomocné funkce.
Následující sekce popisují každou z těchto kategorií a odpovídající nástroje nebo knihovny použité.

%%% Podpodsekce – Kvalita kódu
%%%%% Wording: ⏳
%%%%% Styling: ⏳
%%%%% References: ⏳
%%% --------------------------------------------------------------
\subsubsection{Kvalita kódu}
\label{subsubsec:implementace-techologie-ostatni-kvalita}
Udržování kvality kódu je klíčové v jakémkoli softwarovém projektu, zejména při práci na komplexních aplikacích.
Vysoká kvalita kódu usnadňuje čtení, údržbu a ladění.
Pro zajištění kvality kódu v tomto projektu byly vybrány dva klíčové nástroje: Prettier a ESLint.

\textbf{Prettier}
Prettier je nástroj pro formátování kódu, který vynucuje konzistentní styl napříč celým kódovým základem tím, že kód analyzuje a znovu jej vytiskne podle svých vlastních pravidel.
Integrací Prettier jsou efektivně eliminovány obavy z formátování kódu.
Nástroj automaticky formátuje kód pokaždé, když je uložen, což zajišťuje, že kódová základna je konzistentní ve stylu, bez ohledu na počet přispěvatelů kódu.
To také pomáhá minimalizovat čas strávený diskutováním a opravováním problémů se stylem.

\textbf{ESLint}
ESLint na druhou stranu je nástroj pro statickou analýzu kódu, který identifikuje problematické vzory v kódu JavaScriptu a TypeScriptu.
Zatímco Prettier se stará o problémy s formátováním, ESLint se zaměřuje na hledání potenciálních chyb a problémů s kvalitou kódu.
Integrací ESLint je možné udržovat kvalitu kódu, čímž se kód stává bezpečnějším a předvídatelnějším.

%%% Podpodsekce – Získávání dat
%%%%% Wording: ⏳
%%%%% Styling: ⏳
%%%%% References: ⏳
%%% --------------------------------------------------------------
\subsubsection{Získávání dat}
\label{subsubsec:implementace-techologie-ostatni-ziskavani}
Získávání dat je kritickým aspektem jakékoliv aplikace, která interaguje se serverem.
V tomto projektu byly pro správu stavu serveru použity Axios a React Query.

\textbf{Axios}
Axios je promise-based HTTP klient pro prohlížeč a Node.js, který zjednodušuje odesílání asynchronních HTTP požadavků na REST koncové body.
S funkcemi jako automatická transformace pro JSON data, ochrana klienta proti XSRF a zpracování požadavků a odpovědí zjednodušuje Axios proces získávání dat v aplikaci.

\textbf{React Query}
React Query je knihovna pro synchronizaci dat pro React, která usnadňuje získávání, cachování a aktualizaci stavu serveru v aplikacích React.
Out-of-the-box se stará o cachování, aktualizace na pozadí a zastaralá data bez nutnosti konfigurace.
Použití s Axios zjednodušuje získávání dat a správu stavu, čímž se snižuje potřeba dalších knihoven pro správu stavu.

%%% Podpodsekce – Animace
%%%%% Wording: ⏳
%%%%% Styling: ⏳
%%%%% References: ⏳
%%% --------------------------------------------------------------
\subsubsection{Animace}
\label{subsubsec:implementace-techologie-ostatni-animace}
Animace přidávají život do aplikací a zlepšují uživatelskou zkušenost.
Pro animace v tomto projektu byly použity Framer Motion a React Spring.

\textbf{Framer Motion}
Framer Motion je produkčně připravená knihovna pro animace v Reactu, která poskytuje intuitivní způsob animace prvků.
Zjednodušuje složité animace a umožňuje vytvářet responzivní animace, které mohou interagovat s gesty uživatele.

\textbf{React Spring}
React Spring na druhou stranu je knihovna pro animace, která využívá fyzikálních principů pro vytváření plynulých a přirozených animací.
Pomocí tzv.
springs a dampers animace v React Spring těsně napodobují pohyby ve skutečném světě.

%%% Podpodsekce – Formuláře
%%%%% Wording: ⏳
%%%%% Styling: ⏳
%%%%% References: ⏳
%%% --------------------------------------------------------------
\subsubsection{Formuláře}
\label{subsubsec:implementace-techologie-ostatni-formulare}
Pro správu formulářů v aplikaci byly použity React Hook Form a Zod.

\textbf{React Hook Form}
React Hook Form je knihovna pro správu formulářů v Reactu, která snižuje množství boilerplate kódu potřebného pro vytváření formulářů.
Využitím vestavěných hooků Reactu zjednodušuje React Hook Form validaci formulářů a zlepšuje výkon tím, že minimalizuje počet re-renderů.

\textbf{Zod}
Zod je knihovna pro deklaraci a validaci schémat pro TypeScript a JavaScript.
Použitím s React Hook Form zjednodušuje validaci formulářů a zajišťuje, že data zadaná uživatelem odpovídají definovanému schématu.

%%% Podpodsekce – Pomocné funkce
%%%%% Wording: ⏳
%%%%% Styling: ⏳
%%%%% References: ⏳
%%% --------------------------------------------------------------
\subsubsection{Pomocné funkce}
\label{subsubsec:implementace-techologie-ostatni-pomocne-funkce}
Posledně, Lodash, knihovna pro utility funkce v JavaScriptu, byla použita pro své užitečné funkce jako hluboké klonování objektů, slučování objektů, debouncing a throttling.
To činí kód čistější, čitelnější a udržovatelnější.

Ve shrnutí byly všechny tyto nástroje a knihovny vybrány po pečlivém zvážení jejich funkcí, kompatibility s ostatními částmi technologického stacku a efektivnosti, kterou přinášejí do vývojového procesu.
Hrály významnou roli při vytváření online platformy pro prodej vstupenek a zajistily plynulý a efektivní vývojový proces.
