%%% Sekce – Výběr technologií
%%%%% Wording: ✅
%%%%% Styling: ✅
%%%%% References: ✅
%%%%% Grammar: ✅
%%% --------------------------------------------------------------
\section{Výběr technologií}
\label{sec:implementace-technologie}
Tato kapitola je zaměřena na technologický stack vybraný pro frontendovou implementaci webového řešení prodeje vstupenek s rezervací míst.
Tento výběr není náhodný; každá technologie byla vybrána z určitého důvodu, ať už je to kvůli konkrétní funkčnosti, kterou poskytuje, její synergie s ostatními technologiemi ve stacku, nebo kvůli její robustnosti a spolehlivosti.
Následující sekce podrobně popisují každou z těchto technologií.

%%% Podsekce – React.js
%%%%% Wording: ✅
%%%%% Styling: ✅
%%%%% References: ✅
%%%%% Grammar: ✅
%%% --------------------------------------------------------------
\begin{subsection}{React.js}
    \label{subsec:implementace-technologie-react}
    React.js je JavaScriptová knihovna, která byla zpřístupněna veřejnosti jako \foreign{open-source}\footnote{
        Open-source software je typ software, který je zpřístupněn veřejnosti a jeho zdrojový kód je licencován tak, že umožňuje uživatelům studovat, měnit a distribuovat software kohokoli a za jakýchkoli účelů\cite{rh_open_source_what_is_open_source}.
    } nástroj.
    Byla vytvořena a v současné době je udržována společností Meta (dříve Facebook).
    Tato knihovna pomáhá při vývoji uživatelských rozhraní pro \ac{spa}\footnote{
        \ac{spa} je typ webové aplikace nebo webové stránky, která interaguje s uživateli dynamicky, nahrazuje tradiční model \ac{mpa}\cite{w_blog_spa_vs_mpa}.
    } tím, že umožňuje vývojářům vytvářet opakovaně použitelné komponenty uživatelského rozhraní\cite{fos_projects_react}.
    Výběr React.js pro tento konkrétní projekt je řízen řadou faktorů.

    Zpočátku se knihovna může pochlubit charakteristikou virtuálního \ac{dom}, která zvyšuje rychlost a efektivitu generování složitých uživatelských rozhraní.
    Na rozdíl od konvenčního \foreign{full-refresh}\footnote{\foreign{Full-refresh} je proces, kdy se celá stránka znovu načte, když uživatel provede nějakou akci, která vyžaduje změnu v uživatelském rozhraní.}
    přístupu, který je typický pro \ac{mpa}, umožňuje virtuální \ac{dom} knihovně React omezit přímou manipulaci s \ac{dom}, čímž šetří životně důležité výpočetní zdroje a představuje hladší a robustnější rozhraní\cite{r_docs_faq_internals_what_is_the_virtual_dom}.

    React.js navíc funguje na architektuře založené na komponentách, což umožňuje dekompozici složitých uživatelských rozhraní na menší, opakovaně použitelné komponenty.
    Tato metodika zlepšuje organizaci a údržbu kódu a také zefektivňuje proces ladění a testování\cite{r_learn_describing_the_ui}.
    Tato charakteristika se ukazuje jako obzvláště výhodná v oblasti vývoje komplexních aplikací, kde lze různé prvky uživatelského rozhraní znovu použít ve více částech aplikace.

    React.js těží z robustní komunitní podpory s množstvím zdrojů a nástrojů, které má k dispozici.
    Tato komunální podpora se ukazuje jako neocenitelná při vývoji složitých aplikací, jelikož slouží k urychlení řešení potenciálních problémů.
\end{subsection}

%%% Podsekce – TypeScript
%%%%% Wording: ✅
%%%%% Styling: ✅
%%%%% References: ✅
%%%%% Grammar: ✅
%%% --------------------------------------------------------------
\begin{subsection}{TypeScript}
    \label{subsec:implementace-technologie-typescript}
    TypeScript je programovací jazyk, který je staticky typovanou nadmnožinou JavaScriptu.
    Je vyvinut a spravován společností Microsoft a slouží k rozšíření funkčnosti JavaScriptu a zároveň usnadňuje vytváření složitých aplikací se zlepšenou spolehlivostí a sníženou náchylností k chybám\cite{g_www_geeksforgeeks_org_difference_between_typescript_and_javascript}.

    Přestože je JavaScript robustní programovací jazyk, získal negativní zpětnou vazbu pro svou shovívavost ve svém dynamickém typovém systému.
    TypeScript se snaží tento problém vyřešit implementací statického typování, které minimalizuje chyby za běhu a poskytuje pokročilejší vývojové nástroje, včetně automatického dokončování, odvozování typu a kontroly typu\cite{g_www_geeksforgeeks_org_difference_between_typescript_and_javascript}.

    V poslední době byl TypeScript velmi rozšířen a zpopularizován a stále více se stává normou pro vývoj rozsáhlých JavaScript aplikací\cite{ct_the_relevance_of_typescript_in_2022_}.
    Tento projekt preferuje TypeScript kvůli jeho vynikající spolehlivosti a kompatibilitě s React.js.
\end{subsection}

%%% Podsekce – Next.js
%%%%% Wording: ✅
%%%%% Styling: ✅
%%%%% References: ✅
%%%%% Grammar: ✅
%%% --------------------------------------------------------------
\begin{subsection}{Next.js}
    \label{subsec:implementace-technologie-nextjs}
    Next.js je open-source React framework\footnote{Framework je sada nástrojů, které usnadňují vývoj aplikací.}, který zprvu zjednodušuje proces založení a nastavení nového projektu a zahrnuje spoustu dalších užitečných funkcí v dalších fázích vývoje, jako je \ac{ssr} a \ac{ssg}.
    Byl vytvořen společností Vercel, dříve známou jako ZEIT, a je aktivně udržován a rozvíjen rozsáhlou vývojářskou komunitou.
    Je navržen tak, aby zvýšil výkon a efektivitu React aplikací a vybavil vývojáře nástroji pro tvorbu aplikací s bohatými funkcemi\cite{n_nextjs_org_docs}.

    V kontextu vyvíjené aplikace byl Next.js vybrán pro své funkce, které zjednodušují běžné úkoly.
    Například Next.js má vestavěnou podporu pro \acs{api} endpointy, které umožňují vývojářům vytvářet backend \acs{api} ve stejném repozitáři jako frontend, což je ideální například pro simulaci backendové funkcionality\cite{n_nextjs_org_docs}.

    Next.js dále poskytuje funkce jako file-system routování\footnote{File-system routování je funkce, která umožňuje vývojářům vytvářet strukturu souborů, která se přímo mapuje na URL adresy.}, automatické rozdělení kódu a mnoho dalšího, což snižuje množství potřebné konfigurace a nastavení a umožňuje vývojářům soustředit se na logiku aplikace\cite{n_nextjs_org_docs}.

    V poslední řadě byl Next.js vybrán také pro svůj bezproblematický proces nasazení do cloudu pomocí platformy Vercel, která je vývojářům k dispozici zdarma a umožňuje nasadit aplikaci online během několika málo minut\cite{n_nextjs_org_docs}.
\end{subsection}

%%% Podsekce – Mantine UI
%%%%% Wording: ✅
%%%%% Styling: ✅
%%%%% References: ✅
%%%%% Grammar: ✅
%%% --------------------------------------------------------------
\begin{subsection}{Mantine UI}
    \label{subsec:implementace-technologie-mantine}
    Mantine je komplexní knihovna \ac{ui} komponent pro React s důrazem na použitelnost, přístupnost a \foreign{developer experienece}\footnote{\foreign{Developer experience} označuje kvailtu nástrojů a procesů, které jsou k dispozici pro vývojáře.
    Jedná se o ekvivalent \acl{ux} z pohledu vývojářů.}.
    Zahrnuje řadu komponent, které lze kombinovat a přizpůsobit k vytvoření složitých a vizuálně bohatých uživatelských rozhraní.
    Tato knihovna nabízí jednodušší způsob tvorby uživatelských rozhraní ve srovnání s vytvářením každé komponenty od základů, což může být časově náročné a náchylné k chybám.

    Použitím knihovny komponent, jako je Mantine, se zajistí, že komponenty použité v celé aplikaci jsou konzistentní čímž se nejen velmi snižuje potřeba repetitivního kódu, ale také se zvyšuje přehlednost a čitelnost kódu.
    Dále se díky důrazu Mantine na přístupnost (\foreign{accessibility}) zajišťuje, že aplikace je použitelná co nejširšímu publiku\cite{m__mantine_dev}.

    Díky vynikající dokumentaci a bohatému výběru komponent je Mantine preferovanou volbou pro projekt před jinými podobnými knihovnami, jako je například MaterialUI od společnosti Google\cite{m__mui_com}.
\end{subsection}

%%% Podsekce – Tailwind CSS
%%%%% Wording: ✅
%%%%% Styling: ✅
%%%%% References: ✅
%%%%% Grammar: ✅
%%% --------------------------------------------------------------
\begin{subsection}{Tailwind CSS}
    \label{subsec:implementace-technologie-tailwind}
    Tailwind CSS je \ac{css} framework založený na \foreign{utility-first} přístupu\footnote{\foreign{Utility-first} se zaměřuje na vytváření nízkoúrovňových utility tříd, které lze kombinovat k vytváření složitých rozhraní.
    Jedná se například o třídy jako \mintinline{css}{.text-center} nebo \mintinline{css}{.bg-red-500}.} pro rychlé vytváření vlastních designů.
    Na rozdíl od tradičních CSS frameworků, které poskytují hotové komponenty, umožňuje Tailwind vývojářům vytvářet vlastní designy poskytováním nízkoúrovňových utility tříd, které mohou kombinovat k vytváření jedinečných rozhraní\cite{tc__tailwindcss_com}.

    V rámci vyvíjené aplikace je Tailwind CSS použit především pro snadné stylování rozložení prvků v aplikaci a rychlé prototypování.
    Jeho \foreign{utility-first} přístup zjednodušuje proces tvorby vlastních stylů a jeho responzivní modifikátory umožňují vytvářet designy, které fungují na všech velikostech obrazovky bez nutnosti psaní vlastních \ac{css} tříd\cite{tc__tailwindcss_com}, což značně zrychluje vývoj.
\end{subsection}

%%% Podsekce – Ostatní technologie
%%%%% Wording: ✅
%%%%% Styling: ✅
%%%%% References: ✅
%%%%% Grammar: ✅
%%% --------------------------------------------------------------
\begin{subsection}{Ostatní technologie}
    \label{subsec:implementace-technologie-ostatni}
    Kromě hlavního technologického stacku diskutovaného výše bylo při vývoji aplikace klíčových několik dalších nástrojů a knihoven.
    Každý nástroj či knihovna slouží k jedinečnému účelu a zlepšuje vývojový proces v oblastech jako je kvalita kódu, získávání dat či správa formulářů.
    Následující sekce popisují každou z těchto kategorií a odpovídající nástroje nebo knihovny použité.

%%% Podpodsekce – Kvalita kódu
%%%%% Wording: ✅
%%%%% Styling: ✅
%%%%% References: ✅
%%%%% Grammar: ✅
%%% --------------------------------------------------------------
    \begin{subsubsection}{Kvalita kódu}
        \label{subsubsec:implementace-technologie-ostatni-kvalita}
        Udržování kvality kódu je klíčové v jakémkoli softwarovém projektu, zejména při práci na komplexních aplikacích.
        Vysoká kvalita kódu usnadňuje čtení, údržbu a ladění.
        Pro zajištění kvality kódu v tomto projektu byly vybrány dva klíčové nástroje: \foreign{Prettier}\footnote{\url{https://www.npmjs.com/package/prettier}} a \foreign{ESLint}\footnote{\url{https://www.npmjs.com/package/eslint}}.

        \textbf{Prettier} je nástroj pro formátování kódu, který vynucuje konzistentní styl napříč celým \foreign{codebase}\footnote{\foreign{Codebase} je označení pro celkový kódový základ aplikace.} tím, že kód analyzuje a formátuje podle vlastních pravidel.
        Nástroj automaticky formátuje kód pokaždé, když je například uložen, což zajišťuje, že kód je konzistentní v nastaveném formátu napříč celým projektem\cite{p__prettier_io}.

        \textbf{ESLint} slouží jako nástroj pro analýzu statického kódu speciálně navržený pro detekci problematických vzorů v kódu JavaScript a TypeScript.
        Doplňuje Prettier, protože zatímco Prettier primárně řeší problémy s formátováním, ESLint se primárně zaměřuje na identifikaci potenciálních chyb a problémů s kvalitou kódu.
        Začleněním ESLint do vývojového procesu může být zachována kvalita kódu, což vede ke zvýšené bezpečnosti a předvídatelnosti v celém projektu\cite{e__eslint_org}.
    \end{subsubsection}

%%% Podpodsekce – Získávání dat
%%%%% Wording: ✅
%%%%% Styling: ✅
%%%%% References: ✅
%%%%% Grammar: ✅
%%% --------------------------------------------------------------
    \begin{subsubsection}{Získávání dat}
        \label{subsubsec:implementace-technologie-ostatni-ziskavani}
        Získávání dat hraje zásadní roli v jakékoli aplikaci, která spolupracuje se serverem.
        V tomto projektu bylo pro účely správy komunikace se serverem využito knihoven \foreign{Axios}\footnote{\url{https://www.npmjs.com/package/axios}} a \foreign{React Query}\footnote{\url{https://www.npmjs.com/package/@tanstack/react-query}}.

        \textbf{Axios} je \ac{http} klient, který funguje na základě \foreign{Promise}\footnote{Promise je JavaScriptový objekt, který slouží k reprezentaci asynchronních operací.} a je kompatibilní s prohlížečem i prostředím Node.js\footnote{Node.js je open-source, multiplatformní, JavaScriptové prostředí, které umožňuje spouštět JavaScriptový kód mimo webový prohlížeč\cite{n__nodejs_org}.}.
        Jeho primární funkcí je usnadnit odesílání asynchronních \ac{http} požadavků na backend \ac{api} a zároveň poskytuje užitečné funkce, jako je automatická transformace \ac{json} struktur, ochrana na straně klienta proti \ac{xsrf} a zpracování požadavků/odpovědí.
        Použití Axios může výrazně zlepšit efektivitu a snadnost získávání dat v aplikaci.

        \textbf{React Query} je knihovna navržená pro React, která umožňuje snadnou synchronizaci dat mezi klientem a serverem.
        Poskytuje ukládání do mezipaměti (\foreign{cache}), aktualizace na pozadí a správu zastaralých dat automaticky bez nutnosti jakékoli konfigurace.
        Při použití ve spojení s Axios zjednodušuje načítání dat, což v konečném důsledku snižuje potřebu dalších \foreign{state management}\footnote{\foreign{State management} je proces správy stavu dat aplikace.} knihoven.
    \end{subsubsection}

%%% Podpodsekce – Formuláře
%%%%% Wording: ✅
%%%%% Styling: ✅
%%%%% References: ✅
%%%%% Grammar: ✅
%%% --------------------------------------------------------------
    \begin{subsubsection}{Formuláře}
        \label{subsubsec:implementace-technologie-ostatni-formulare}
        Pro správu formulářů v aplikaci byly použity knihovny \foreign{React Hook Form}\footnote{\url{https://www.npmjs.com/package/react-hook-form}} a \foreign{Zod}\footnote{\url{https://www.npmjs.com/package/zod}}.

        \textbf{React Hook Form} je knihovna pro správu formulářů v Reactu, která snižuje množství \foreign{boilerplate}\footnote{\foreign{Boilerplate} je označení pro kód, který je nezbytný, aby bylo možné vykonat určitou akci.
        V tomto případě se jedná o nezbytný kód pro vytváření formulářů.} kódu potřebného pro vytváření formulářů.
        Využitím vestavěných hooků\footnote{Hook je označení pro speciální React funkce, které lze využít v rámci životního cyklu komponenty\cite{fos_projects_react}.} Reactu zjednodušuje validaci formulářů a zlepšuje výkon tím, že minimalizuje počet \foreign{re-renderů}\footnote{\foreign{Re-render} je proces znovuvykreslení komponenty, nejčastěj se děje na základě změny stavu komponenty.}.

        \textbf{Zod} je knihovna pro deklaraci a validaci schémat pro TypeScript a JavaScript.
        Díky Zod lze deklarovat schémata a validovat je pomocí vestavěných funkcí.
        Použitím s React Hook Form zjednodušuje validaci formulářů a zajišťuje, že data zadaná uživatelem odpovídají definovanému schématu.
    \end{subsubsection}
\end{subsection}

%%% Podsekce – Správa zdrojového kódu
%%%%% Wording: ✅
%%%%% Styling: ✅
%%%%% References: ✅
%%%%% Grammar: ✅
%% --------------------------------------------------------------
\begin{subsection}{Správa zdrojového kódu}
    \label{subsec:implementace-sprava-zdrojoveho-kodu}
    Efektivní správa zdrojového kódu je nedílnou součástí vývoje softwaru, zejména v kolaborativních prostředích.
    Git, bezplatný a open-source distribuovaný systém pro správu verzí, je nejrozšířenějším nástrojem pro správu zdrojového kódu.
    Git umožňuje vývojářům efektivně sledovat a spravovat změny v kódu, což usnadňuje plynulý vývojový proces\cite{g__git_scm_com}.

    Nicméně Git sám o sobě nenabízí kompletní řešení pro správu projektu.
    \textbf{GitHub} doplňuje Git poskytováním webového rozhraní pro hostování a spolupráci na Git repozitářích.
    Mezi funkce GitHubu patří řízení přístupu, správa úkolů, sledování chyb, požadavků na funkce a další\cite{g_github_com_about}.
    To z něj činí nezbytný nástroj pro správu a sledování průběhu projektů.
    Mimo jiné, GitHub nabízí také prémiový plán pro vzdělávací instituce, který umožňuje studentům a učitelům využívat všechny funkce GitHubu zdarma\cite{e_education_github_com_pack}.

    Vyvíjený prototyp byl tedy hostován prostřednictvím GitHub a je dostupný na adrese \url{https://github.com/filipditrich/bachelor-thesis-project}.

    Kromě toho, bylo využito konvenčního zápisu \foreign{commit zpráv}\footnote{\foreign{Commit} je označení pro jednotku změny v Gitu. \foreign{Commit zpráva} je zpráva, která popisuje změny provedené v rámci \foreign{commitu}.}, standardizovaného stylu pro commit zprávy, který zvyšuje čitelnost a pomáhá generovat poznámky k vydání, označované jako \foreign{release notes}.
    Tato konvence specifikuje strukturu commit zpráv, včetně typu (feat, fix, chore, docs atd.), volitelného rozsahu a stručného popisu.
    Například commit zpráva může mít podobu \texttt{feat(lib/feature): add cart feature, implement useCart hook to manage complex shopping cart state}, kde \texttt{feat} označuje novou funkcionalitu, \texttt{lib/feature} označuje oblast změny a \texttt{add cart feature, implement useCart hook to manage complex shopping cart state} je stručný popis změny.
    Tento příklad je z reálného commitu (\texttt{3982b9a}) vytvořeného v rámci vývoje projektu.
\end{subsection}
