%%% Kapitola - Implementace aplikace
%%%%% Wording: ✅
%%%%% Styling: ✅
%%%%% References: ✅
%%% --------------------------------------------------------------
\chapter{Implementace aplikace}
\label{ch:implementace}
Vývoj frontendu je zásadním aspektem interakce uživatele s aplikací.
Znamená to navrhnout vizuálně atraktivní a uživatelsky přívětivé rozhraní, které je také efektivní a přizpůsobitelné.
Tato část se hluboce ponoří do specifik vývoje frontendu pro vyvíjenou aplikaci pro prodej vstupenek s rezervací míst a poskytuje komplexní informace o vybraných technologiích, architektuře projektu, klíčových funkcích a jejich implementaci.

Úvodní kapitola poskytuje vysvětlení pro volbu určitých technologií pro frontend, jako jsou React.js, TypeScript, Next.js, Mantine UI, Tailwind CSS a další nástroje.
Odůvodnění začlenění těchto technologií a jejich příslušných funkcí do aplikace bude rozvedeno v části~\ref{sec:implementace-technologie}.

V části~\ref{sec:implementace-architektura} je uveden přehled struktury a architektury projektu po vysvětlení technologického stacku\footnote{Technologický stack je soubor softwarových technologií a nástrojů, které jsou používány vývojáři pro vývoj softwaru nebo webových stránek.}.
Tato část se bude zabývat postupy provedenými k zahájení projektu a rámcem navrženým pro usnadnění fungování této komplexní aplikace.

Oddíly~\ref{sec:implementace-seating} až~\ref{sec:implementace-checkout} dokumentu se ponoří do základních funkcí aplikace, konkrétně do interaktivní mapy sedadel, správy nákupního košíku, rezervačního systému a procesu vyřízení objednávky.
Tyto části poskytují podrobnou analýzu procesu implementace pro každou z těchto funkcí, včetně problémů, se kterým bylo v ramci vývoje čeleno, a strategií použitých k jejich překonání.

Část~\ref{sec:implementace-zaver} představuje komplexní závěr, který shrnuje celý proces vývoje frontendu.
Tato část zdůrazňuje zásadní poznatky, úspěchy a konečnou perspektivu aplikace.
