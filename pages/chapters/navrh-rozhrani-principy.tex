%%% Sekce - Principy návrhu uživatelského rozhraní
%%%%% Wording: ✅
%%%%% Styling: ✅
%%%%% References: ✅
%%%%% Grammar: ✅
%%% --------------------------------------------------------------
\section{Principy návrhu uživatelského rozhraní}
\label{sec:navrh-uzivatelskeho-rozhrani-principy}
Návrh uživatelského rozhraní je poměrně rozsáhlá disciplína, která se zaměřuje na vizuální a interaktivní aspekty produktu.
Při návrhu \ac{ui} je důležité dodržovat určité principy, které zajišťují optimální uživatelskou zkušenost.
Tato sekce shrnuje některé základní principy návrhu \ac{ui} a posuzuje jejich implikace v kontextu aplikace pro prodej vstupenek s rezervací míst.

\textbf{Jednoduchost}\\
Návrh \ac{ui} by měl směřovat k jednoduchosti.
Čím méně úsilí musí uživatel vynaložit na pochopení rozhraní, tím lepší bude celková uživatelská zkušenost.
Čisté, jednoduché rozhraní s jasným zaměřením na funkčnost snižuje kognitivní zátěž a zvyšuje použitelnost\cite{d_resources_ui_design_principles}.

\textbf{Konzistence}\\
Tento princip prosazuje zachování jednotnosti napříč všemi prvky \ac{ui}.
Konzistence se projevuje v použití podobných prvků, akcí a designu napříč celým rozhraním\cite{d_resources_ui_design_principles}.
Například pokud určitá barva značí interaktivní prvek na plánu sedadel, stejná barva by měla být použita i pro značení interaktivních prvků jinde v rámci aplikace.
Tímto se zvyšuje předvídatelnost, což uživatelům usnadňuje orientaci a navigaci v rozhraní.

\textbf{Pocit kontroly}\\
Základním principem návrhu \ac{ui} je umožnit uživateli cítit se vždy v kontrole nad produktem.
Toho lze dosáhnout návrhem transparentního a intuitivního systému, ve kterém uživatel vždy ví, kde se nachází a jak postupovat\cite{d_resources_ui_design_principles}.
V kontextu aplikace pro prodej vstupenek to může znamenat poskytnutí jasného a zřejmého způsobu, jak uživatelé mohou přejít k výběru sedadla, přidání do košíku a dokončení objednávky.

\textbf{Zpětná vazba}\\
Zpětná vazba je klíčovým aspektem každé interakce, protože potvrzuje nebo informuje uživatele o vykonaných akcích\cite{d_resources_ui_design_principles}.
Vizuální indikátory, jako je zvýraznění vybraného sedadla nebo potvrzovací zpráva při přidání vstupenky do košíku, poskytují uživateli okamžitou zpětnou vazbu.
Tím se snižuje nejistota a zvyšuje se důvěra uživatele v rozhraní.

\textbf{Prevence a řešení chyb}\\
Chyby jsou nevyhnutelné v jakékoli interakci, ale dobře navržené \ac{ui} může zabránit většině uživatelských chyb nebo zjednodušit jejich řešení\cite{d_resources_ui_design_principles}.
To může znamenat například zakázání tlačítka \textit{Pokračovat}, dokud není vybráno sedadlo nebo zobrazení jasných a užitečných chybových zpráv, když něco selže.

\textbf{Afordance a signifikance}\\
\foreign{Afordance} se vztahuje k vlastnosti objektu, která naznačuje, jak se má používat.
\foreign{Signifikance} jsou vizuálními nápovědami k těmto \foreign{afordancím}\cite{w_topics_affordances}.
Například sedadlo na plánu sedadel může být navrženo tak, aby naznačovalo, že na něj lze kliknout (\foreign{afordance}), a změna kurzoru při najetí na sedadlo (\foreign{signifikance}) může tuto zprávu posílit.

Pochopení a aplikace těchto základních principů návrhu \ac{ui} je klíčové pro vytvoření intuitivního a uživatelsky přívětivého rozhraní.
Tyto principy řídí řadu rozhodnutí v rámci návrhu a pomáhají návrhu \ac{ui} s celkovým cílem poskytnout uživatelům bezproblémový zážitek z rezervace vstupenek.
Další sekce se zabývá tím, jak lze hierarchii Maslowa aplikovat pro další zlepšení návrhu zaměřeného na uživatele.
