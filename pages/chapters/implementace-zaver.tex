%%% Sekce – Závěr
%%%%% Wording: ⏳
%%%%% Styling: ⏳
%%%%% References: ⏳
%%% --------------------------------------------------------------
\section{Závěr}
\label{sec:implementace-zaver}
V této kapitole byl poskytnut podrobný popis procesu a důvodů za rozhodnutím a implementací aplikace v Reactu s TypeScriptem pro systém pro prodej vstupenek na koncerty.
Ukazuje složitou cestu vytváření komplexní frontendové aplikace od základu, přičemž každý krok je podrobně vysvětlen.

Počáteční fáze zahrnovala rozhodnutí o technologickém stacku, který upřednostňuje sílu Reactu a TypeScriptu pro vytvoření škálovatelné a udržovatelné aplikace.
Architektura a struktura projektu byly metodicky uspořádány podle moderních nejlepších postupů s ohledem na potenciální růst projektu.

Čtyři klíčové funkce - interaktivní mapy sedadel, správa košíku, rezervační systém a proces pokladny - byly zkoumány, aby poskytly komplexní pochopení mechanismů, které byly do jejich implementace vloženy.
Tyto funkce se s důrazem na uživatelský zážitek snaží zpříjemnit a zjednodušit proces rezervace vstupenek.

Interaktivní mapa sedadel tvoří srdce aplikace, ilustruje dynamické použití manipulace s SVG a manipulaci s událostmi, aby vytvořila angažované rozhraní pro výběr sedadel.
Následující sekce podrobně popisují správu nákupního košíku, rezervační systém a proces pokladny.
Nákupní košík slouží jako most mezi výběrem sedadel a pokladnou, kterou spravuje globální kontext, aby poskytoval uživatelský zážitek bez přerušení, zatímco rezervační a pokladní procesy nabízejí přímou cestu uživatelům k rezervaci a nákupu požadovaných vstupenek.

Je důležité poznamenat, že tato aplikace slouží jako prototyp, který se integruje s falešným API, a další reálné zvažování, jako je zabezpečení, optimalizace výkonu a rozsáhlé zpracování chyb, může vyžadovat další pozornost v aplikaci připravené pro produkci.

Souhrnně tato kapitola ilustruje, jak moderní technologie jako React a TypeScript, spolu s promyšleným plánováním a provedením architektury, mohou být kombinovány pro vytvoření interaktivní a funkční webové aplikace.
Tento systém pro prodej vstupenek stojí jako důkaz o síle a flexibilitě, které tyto technologie nabízejí, řešení složitých problémů a vytváření angažovaných uživatelských zážitků.

% TODO: final screenshots

Následující kapitoly pojednávají o problémech, které se vyskytly během tohoto projektu, o použitých strategiích k jejich překonání a o získaných zkušenostech.
Reflexe této cesty poskytuje cenné poznatky a vodítko pro budoucí projekty podobného charakteru.
