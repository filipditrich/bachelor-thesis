\chapter{Praktická část}

Praktická část pojednává o~vývoji prototypu frontendu webové aplikace pro prodej vstupenek s~důrazem na implementaci funkčnosti rezervace míst. Nutno podotknout že výsledný prototyp nebude a ani není v plánu, aby byl plně funkční, nýbrž pouze ukazuje možnou implementaci konrkétních zvolených částí.\\

K implementaci prototypu je důležité předem vydefinovat jasnou specifikaci a požadavky na výsledný produkt. Bez těchto speicifkací by nebylo možné finální výsledek objektivně zhodnotit. Po jasné specifiaci požadavků bude potřeba prototyp vizuálně navrhnout a připravit jako podklad k implementaci. K té bude také třeba zanalyzovat konkrétní požadavky a zvolit správné technologie. Tento prototyp bude realizován pouze z frontendové části, tedy z pohledu vizuálního rozhraní pro potenciálního zákazníka, který si bude chtít zakoupit vstupenku s využitím rezervace místa. Z tohoto důvodu bude alespoň minimálně popsána funkčnost dostupného backednového rozhraní, který bude sloužit jako zdroj dat pro frontend.\\

Všechny zmíněné postupy budou v této části práce blíže popsány a vysvětleny v jednotlivých kapitolách.

\section{Specifikace a požadavky}
\label{sec:specifikace}
TODO

\section{Vizuální návrh}
\label{sec:navrh}
TODO

\section{Výběr technologií}
\label{sec:technologie}
TODO
