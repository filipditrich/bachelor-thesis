%%% Sekce - Úvod do vývoje frontendu
%%%%% Wording: ⏳
%%%%% Styling: ⏳
%%%%% References: ⏳
%%% --------------------------------------------------------------
\section{Úvod do vývoje frontendu}
\label{sec:implementace-uvod}
Frontendový vývoj, hovorově známý jako vývoj na straně klienta, je nezbytnou součástí vývoje webových aplikací, která se zaměřuje na uživatelské rozhraní a uživatelskou zkušenost.
Týká se všeho, s čím uživatelé vizuálně interagují, včetně návrhu, struktury, chování a obsahu webové aplikace.

Frontend webové aplikace je v zásadě postaven na trojici technologií, a to HTML (HyperText Markup Language), CSS (Cascading Style Sheets) a JavaScript.
HTML strukturuje obsah na webové stránce, CSS jej stylově upravuje a JavaScript jej dělá interaktivní.

Nicméně, oblast frontendového vývoje sahá mnohem dále než tato základní trojice.
V současném vývoji webových aplikací frontendoví vývojáři často využívají sílu pokročilých JavaScriptových knihoven a frameworků, jako jsou React.js, Angular, Vue.js a další, pro vytváření sofistikovaných a interaktivních uživatelských rozhraní.
Kromě toho jsou k dispozici CSS frameworky, jako je Bootstrap nebo Tailwind CSS, které poskytují předdefinované třídy, aby urychlily proces stylování.

Úloha frontendového vývojáře se neomezuje na implementaci návrhů nebo strukturování obsahu; jsou také zodpovědní za to, aby aplikace byla responzivní, výkonná a přístupná.
To znamená, že webová stránka by měla bezchybně fungovat na různých zařízeních s různými velikostmi obrazovek a měla by dodržovat nejlepší postupy v oblasti výkonu, aby poskytovala uživatelům plynulý zážitek.
Kromě toho by aplikace měla být přístupná a měla by uspokojovat uživatele s různými schopnostmi a preferencemi.

Zatímco backendový vývoj se týká serveru a toho, jak aplikace funguje, frontendový vývoj se týká toho, jak aplikace vypadá a jak se uživatel cítí.
Jde o umění a vědu tvorby uživatelsky přívětivého rozhraní, které reaguje na uživatelské interakce intuitivním a angažovaným způsobem.

V následujících sekcích budou představeny a diskutovány konkrétní technologie vybrané pro frontendový vývoj platformy pro prodej vstupenek, přičemž bude osvětlen jejich konkrétní role a důvody jejich výběru.
