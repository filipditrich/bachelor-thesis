%%% Sekce - Úvod do vývoje frontendu
%%%%% Wording: ✅
%%%%% Styling: ✅
%%%%% References: ✅
%%%%% Grammar: ✅
%%% --------------------------------------------------------------
\section{Úvod do vývoje frontendu}
\label{sec:implementace-uvod}
Vývoj frontendu, běžně označovaný jako vývoj na straně klienta, představuje klíčovou součást vývoje webu, která se soustředí na vizuální rozhraní a uživatelskou zkušenost.
Zahrnuje všechny aspekty webové aplikace, se kterou se uživatelé vizuálně zabývají, jako je její design, rozvržení, ovládání a obsah.

Základní komponenty používané při vytváření frontendu webových aplikací jsou \ac{html}, \ac{css} a JavaScript.
\ac{html} je odpovědné za strukturování obsahu webové stránky, \ac{css} se používá k aplikaci stylů na webovou stránku a JavaScript se používá k vytváření interaktivních prvků na webové stránce\cite{mdn_getting_started_with_the_web}.

Oblast vývoje frontendu přesahuje základní tři komponenty.
V moderním vývoji webových aplikací využívají vývojáři frontendu sofistikované knihovny a frameworky JavaScriptu, jako jsou \textbf{React.js}, \foreign{Angular}, \foreign{Vue.js}, mimo jiné, k vytváření složitých a interaktivních uživatelských rozhraní\cite{mdn_tools_and_testing_client_side_javascript_frameworks_introduction}.
\ac{css} knihovny jako \foreign{Bootstrap} nebo \textbf{Tailwind CSS} navíc nabízejí předdefinované třídy pro urychlení procesu stylování\cite{bs_guide_top_css_frameworks}.

Odpovědnosti frontendového vývojáře přesahují pouhou implementaci návrhu a organizaci obsahu.
Zahrnují klíčové úkoly zajištění odezvy, výkonu a dostupnosti aplikace.
To znamená zajistit, aby webová stránka bez problémů fungovala na řadě zařízení s různou velikostí obrazovky, a zároveň dodržovat strategie optimalizace výkonu, aby byla zajištěna bezproblémová uživatelská zkušenost (\ax{ux}).
Kromě toho musí být aplikace navržena tak, aby vyhovovala uživatelům s různými schopnostmi a preferencemi\cite{mdn_front_end_web_developer}.

Vývoj backendu se zaměřuje na server a funkčnost aplikace, zatímco vývoj frontendu se soustředí na vizuální a interaktivní aspekty aplikace, které zažívá koncový uživatel.
Zahrnuje dovednou kombinaci estetiky a použitelnosti k vytvoření rozhraní, které je intuitivní a podmanivé svou odezvou na interakce uživatele\cite{mdn_server_side_first_steps_introduction}.

V následujících částech budou představeny a analyzovány vybrané technologie pro vývoj frontendu vyvíjené aplikace, objasní jejich jednotlivé funkce a zdůvodnění jejich použití.
