%%%%% Úvod
%%%%% ------------------------------------------------------------
\chapter*{Úvod}
\addcontentsline{toc}{chapter}{Úvod}

%%% Sekce - Prodej vstupenek
%%% --------------------------------------------------------------
\section*{Prodej vstupenek}
\addcontentsline{toc}{section}{Prodej vstupenek}
\label{sec:uvod-prodej-vstupenek}
Prodej vstupenek na kulturní a jiné různé události je důležitou součástí zábavního průmyslu, neboť poskytuje lidem přístup na koncerty, divadelní představení, sportovní či jiné události. Prodej vstupenek umožňuje pořadatelům těchto akcí nejen kontrolovaný průběh akce, ale především generuje dostatečný finanční tok peněz před konáním jejich akce. Tyto finance zpravidla potřebují pro zajištění všech potřebných prostředků pro uspořádání akce a pro pořadatele se tedy jedná o jeden z klíčových faktorů úspěchu konání akce. Potřebují tedy pro zákazníky zajistit co nejsnadnější a nejpříjemnější možnost nákupu vstupenek.\\

S nástupem moderních technologií se online prodej vstupenek proměnil v atraktivní a preferovaný způsob nákupu vstupenek, nabízející zákazníkům snadný, pohodlný a hlavně rychlý způsob nákupu vstupenek, aniž by se museli kamkoliv fyzicky dostavit. Tento nový moderní způsob prodeje vstupenek však nabízí výhody nejen zákazníkům, ale také pořadatelům akcí. Sytémy, které jsou na tomto způsobu prodeje vstupenek založeny, pořadatelům akcí umožňují bezproblémový prodej vstupenek, což vede k efektivnějšímu plánování a řízení akcí. Tyto systémy pořadatelům také nabízejí cenné údaje o zákazních, jejích preferencí a chování, které mohou využít při plánování marketingových strategií, cílených reklam či propagačních akcí za účelem zvýšení zapojení zákazníků a podpoření prodeje.\\

Jedním z nejvýznamnějším pokrokem v této oblasti online řešení prodeje vstupenek bylo rozšíření o možnost rezervace míst v prostoru konání akce. Toto řešení nově zákazníkům umožňuje zarezervovsat si místo na dané udáosti, což opět přináší několik výhod pro zákazníky, ale také pro pořadatele akcí. Zákazníkům umožňuje rezervaci a výběr místa, které je pro ně nejvhodnější. Pořadatelům akcí pak rezervace míst umožňuje předem plánovat kapacitu dané akce a také zjistit, jaké místo je pro zákazníky nejvíce preferované. Dále také značně snižuje počet možných podvodů se vstupenkami, jelikož kapacita je jasně dána počtem míst k sezení a nelze ji snadno překročit.\\

Webová řešení prodeje vstupenek s rezervací míst se v posledních letech stávají stále více oblíbenými a využívanými v různých odvětvích, včetně zábavního průmyslu, sportu či cestování. Avšak s rapdiním vývojem v oblasti webových technoloigí je důležité sledovat a využívat nové trendy a technologie a přizpůsobovat jim takováto řešení, aby byla pro zákazníky stále atraktivní a relevantní. Tato práce se proto zaměřuje na vývoj frontendové části webové aplikace prodeje vstupenek s rezervací míst, která bude využívat moderní webové technologie a nástroje, které jsou v současné době nejvíce využívané a oblíbené.

%%% Sekce - Cíle práce
%%% --------------------------------------------------------------
\section*{Cíle práce}
\addcontentsline{toc}{section}{Cíle práce}
\label{sec:uvod-cile-prace}
Cílem této práce je vyvinout prototyp responzivní webové aplikace nabízející prodej vstupenek s rezervací míst se zaměřením převážně na vývoj frontendové části. Výsledkem této práce bude webová aplikace vyvinuta moderními webovými nástroji a technologiemi, která umožní potenciálním zákazníkům zobrazit mapu areálu nějaké akce či kulturní události, vybrat si jedno či více preferovaných míst, přidat si vstupenky do nákupního košíku a vytvořit tak objednávku. Tato práce se bude zabývat vývojem takového webového řešení, ale pouze z pohledu frontendové části. Ostatní funkčnosti, jako například backednový systém či administrační řešení, nebudou součástí této práce.\\

Práce je strukturovaná do několika částí, počínající touto úvodní kapitolou, která stručně shrnula pozadí trhu online prodeje vstupenek a systémů rezervací míst. Následující kapitola poté takovéto online systémy na tamním trhu poskytovatelů zanalyzuje, a to převážně z pohledu jejich strategií, taktit, silných a slabých stránek v oblasti online prodeje vstupenek a systémů rezervací míst.\\

Dále se práce bude zabývat nejčastějšími technickými výzvami, které se mohou při vývoji takového systému vyskytnout. Opět půjde o technické výzvy převážně z pohledu frontendového řešení, jako například zajištění plné responzivity, implementace optimalizovaného plánku sedaček, zajištění dostupnosti dat v reálném čase, implementace bezpečného a spolehlivého systému rezervací míst a další. Tyto vývzy budou v podrobnosti vysvětleny a budou navržena možná řešení, která by mohla být použita při vývoji takového systému.\\

Praktická část práce se bude převážně zabývat vývojem frontendové části webové aplikace umožňující nákup vstupenek s rezervací místa. V úvodní části bude vymezen rozsah funkčnosti aplikace a bude uveden podrobný popis hlavních funkcí, které bude aplikace nabízet. V této části budou také stručně uvedeny další funkčnosti, které ale nespadají do rámce této práce a bude vysvětleno proč byly záměrně vynechány. Důraz bude kladen na jasnou specifikaci a požadavky, které musí být splněny, aby bylo možné výsledný prototyp objektivně zhodnotit.\\

Následující kapitola bude pojednávat o grafickém návrhu uživatelského rozhraní aplikace. V této části budou také stručně shrnuty návrhové vzory a nejlepší praktiky při návrhu uživatelských rozhraní, v kontextu navrhovaného prototypu, které zajišťují efektivní a uživatelsky přívětivé zážitky. Cílem této části je vytvořit grafický návrh uživatelského rozhraní aplikace, který bude zároveň přehledný, intuitivní a přívětivý pro uživatele. A který bude zároveň použit v části implementace, jako předloha vyvíjené aplikace.\\

Hlavní kapitola praktické části se bude zabývat implementací frontendové části webové aplikace. V této části bude vytvořen prototyp webové aplikace, který bude splňovat všechny požadavky a specifikace, které byly v úvodní části práce vyspecifikovány. Dle analýzy těchto požadavaků budou nejprve vybrány vhodné nástroje a technologie, které budou při vývoji použity. Tyto nástroje a technologie budout také ve stručnosti představeny a bude vysvětlen důvod jejich výběru. Dále se bude tato kapitola zabývat postupem vývoje takovéto webové aplikace od začátku až do konce s důrazem na popis a implementaci hlavní komponenty zajišťující vykreslení interaktivního plánku sedaček v areálu. Tato kapitola bude zakončena technickou dokumentací jednotlivých komponent, které byly vytvořeny při vývoji aplikace, jejich komunikaci a využití.\\

V závěru bude výsledný prototyp vyhodnocen a porovnán s požadavky a specifikacemi, které byly v úvodní části práce vyspecifikovány. Dále budou také uvedena další možná vylepšení a rozšíření, která by mohla být v budoucnu v~aplikaci implementována.
