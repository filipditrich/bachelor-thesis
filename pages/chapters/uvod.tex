%%%%% Úvod
%%%%% ------------------------------------------------------------
\chapter*{Úvod}
\addcontentsline{toc}{chapter}{Úvod}

%%% Sekce - Prodej vstupenek
%%%%% Wording: ✅
%%%%% Styling: ✅
%%%%% References: ✅
%%% --------------------------------------------------------------
\section*{Prodej vstupenek}
\addcontentsline{toc}{section}{Prodej vstupenek}
\label{sec:uvod-prodej-vstupenek}
Prodej vstupenek na kulturní a jiné různé události je důležitou součástí zábavního průmyslu, neboť poskytuje lidem přístup na koncerty, divadelní představení, sportovní či jiné události.
Prodej vstupenek umožňuje pořadatelům těchto akcí nejen kontrolovaný průběh akce, ale především generuje dostatečný finanční tok peněz před konáním jejich akce.
Tyto finance zpravidla potřebují pro zajištění všech potřebných prostředků pro uspořádání akce a pro pořadatele se tedy jedná o jeden z klíčových faktorů úspěchu konání akce.
Potřebují tedy pro zákazníky zajistit co nejsnadnější a nejpříjemnější možnost nákupu vstupenek.

S nástupem moderních technologií se online prodej vstupenek proměnil v atraktivní a preferovaný způsob nákupu vstupenek, nabízející zákazníkům snadný, pohodlný a hlavně rychlý způsob nákupu vstupenek, aniž by se museli kamkoliv fyzicky dostavit.
Tento nový moderní způsob prodeje vstupenek však nabízí výhody nejen zákazníkům, ale také pořadatelům akcí.
Sytémy, které jsou na tomto způsobu prodeje vstupenek založeny, pořadatelům akcí umožňují bezproblémový prodej vstupenek, což vede k efektivnějšímu plánování a řízení akcí.
Tyto systémy pořadatelům také nabízejí cenné údaje o zákazních, jejích preferencí a chování, které mohou využít při plánování marketingových strategií, cílených reklam či propagačních akcí za účelem zvýšení zapojení zákazníků a podpoření prodeje.

Jedním z nejvýznamnějším pokrokem v této oblasti online řešení prodeje vstupenek bylo rozšíření o možnost rezervace míst v prostoru konání akce.
Toto řešení nově zákazníkům umožňuje zarezervovsat si místo na dané udáosti, což opět přináší několik výhod pro zákazníky, ale také pro pořadatele akcí.
Zákazníkům umožňuje rezervaci a výběr místa, které je pro ně nejvhodnější.
Pořadatelům akcí pak rezervace míst umožňuje předem plánovat kapacitu dané akce a také zjistit, jaké místo je pro zákazníky nejvíce preferované.
Dále také značně snižuje počet možných podvodů se vstupenkami, jelikož kapacita je jasně dána počtem míst k sezení a nelze ji snadno překročit.

Webová řešení prodeje vstupenek s rezervací míst se v posledních letech stávají stále více oblíbenými a využívanými v různých odvětvích, včetně zábavního průmyslu, sportu či cestování.
Avšak s rapdiním vývojem v oblasti webových technoloigí je důležité sledovat a využívat nové trendy a technologie a přizpůsobovat jim takováto řešení, aby byla pro zákazníky stále atraktivní a relevantní.
Tato práce se proto zaměřuje na vývoj frontendové části webové aplikace prodeje vstupenek s rezervací míst, která bude využívat moderní webové technologie a nástroje, které jsou v současné době nejvíce využívané a oblíbené.

%%% Sekce - Cíle práce
%%%%% Wording: ✅
%%%%% Styling: ✅
%%%%% References: ✅
%%% --------------------------------------------------------------
\section*{Cíle práce}
\addcontentsline{toc}{section}{Cíle práce}
\label{sec:uvod-cile-prace}
Cílem této práce je vyvinout prototyp responzivní webové aplikace nabízející prodej vstupenek s rezervací míst se zaměřením převážně na vývoj frontendové části.
Výsledkem této práce bude webová aplikace vyvinuta moderními webovými nástroji a technologiemi, která umožní potenciálním zákazníkům zobrazit mapu areálu nějaké akce či kulturní události, vybrat si jedno či více preferovaných míst, přidat si vstupenky do nákupního košíku a vytvořit tak objednávku.
Tato práce se bude zabývat vývojem takového webového řešení, ale pouze z pohledu frontendové části.
Ostatní funkčnosti, jako například backednový systém či administrační řešení, nebudou součástí této práce.

Práce je strukturovaná do několika částí, počínající touto úvodní kapitolou, která stručně shrnula pozadí trhu online prodeje vstupenek a systémů rezervací míst.
Následující kapitola pak identifikuje klíčové části takovýchto systémů a popisuje jejich funkčnost demonstrovanou na příkladu existujících řešení na trhu.

Praktická část práce se bude převážně zabývat vývojem frontendové části webové aplikace umožňující nákup vstupenek s rezervací místa.
Úvodní část bude zaměřena na problematiku návrhu uživatelských rozhraní, definice uživatelských příběhů a následné vytvoření grafického návrhu uživatelského rozhraní aplikace.
Tento grafický návrh dále poslouží jako základ pro implementační část práce.

Hlavní kapitola praktické části se bude zabývat implementací frontendové části webové aplikace.
Kapitola nejprve poskytne úvod do vývoje frontedových aplikací a následně se bude zabývat výběrem technologií a nástrojů, které budou použity při vývoji aplikace.
Dále bude popsána struktura a architektura aplikace, která se dále bude rozvíjet a implementovat.
Klíčové prvky, jako interaktivní mapa sedadel či nákupní košík budou v rámci této kapitoly podrodbněji popsány a bude vysvětlena jejich implementace.
Závěr kapitoly bude věnován procesu nasazení aplikace do produkčního prostředí.

V neposlední řadě se práce zaměří na identifikaci zajímavých problémů, které se vyskytly při vývoji aplikace a jejich následné řešení.
V závěru bude výsledný prototyp aplikace zhodnocen a budou navržena možná vylepšení a rozšíření, která by mohla být v budoucnu v aplikaci implementována.
