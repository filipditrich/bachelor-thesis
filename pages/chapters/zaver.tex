%%% Kapitola – Závěr
%%%%%%% Wording: ✅
%%%%%%% Styling: ✅
%%%%%%% References: ✅
%%%%% Grammar: ✅
%%% --------------------------------------------------------------
\chapter*{Závěr}
\addcontentsline{toc}{chapter}{Závěr}
\label{ch:zaver}

%%% Sekce – Shrnutí
%%%%%%% Wording: ✅
%%%%%%% Styling: ✅
%%%%%%% References: ✅
%%%%% Grammar: ✅
%%% --------------------------------------------------------------
\begin{section}{Shrnutí}
    \label{sec:zaver-shrnuti}
    Cesta této práce začala s jasným cílem důkladně prozkoumat, vyvinout a nasadit výjimečný systém pro prodej vstupenek s pozorností třem zásadním prvkům: interaktivní mapě sedadel, bezproblémovému nákupnímu košíku a zjednodušenému procesu vyřízení objednávky.

    První část práce se zaměřila na identifikaci klíčových částí takovýchto systémů se zvláštním důrazem na jejich uživatelské rozhraní.
    Jednou z těchto esenciálních částí je interaktivní mapa sedadel, která umožňuje uživatelům vybrat si místa, která chtějí zakoupit.
    Důraz byl kladen na vytvoření uživatelsky přívětivého rozložení mapy, její struktury, optimalizovaného barevného kódování pro lepší vizuální orientaci, zavedení intuitivních ovládacích prvků a zajištění plynulého uživatelského zážitku.

    Další zkoumanou částí byl nákupní košík – základní součást jakéhokoliv prodejního systému.
    Tato sekce se zaměřila na datovou správu obsahu košíku, a především na vytvoření uživatelsky přívětivého rozhraní, které umožňuje uživatelům snadno přidávat a odebírat položky z košíku.
    Cílem bylo vytvořit košík, který je intuitivní, snadno použitelný a zároveň poskytuje uživatelům dostatek informací o jejich objednávce.

    Poslední zkoumanou částí byl proces vyřízení objednávky, zaměřující se na zjednodušení a zefektivnění procesu objednávání vstupenek.
    Zaměřením této sekce bylo opět vytvoření rozhraní, které umožňuje uživatelům snadno zadat své osobní údaje, zvolit způsob platby a dokončit objednávku.
    Důraz byl také kladen nejen na proces dokončení objednávky, ale také na její následné zpracování a potvrzení.

    Následně se práce ponořila do světa návrhu \ac{ui}, kritické části rozhodujícího o úspěchu jakékoliv aplikace.
    Tato sekce diskutovala o principech návrhu, jejich významu a návaznosti na psychologii uživatelů, zejména z pohledu, jak může být využita Maslowova pyramida potřeb v návrhu uživatelských rozhraní.
    Důležitým tématem bylo porozumění uživatelským příběhům, jejich efektivnímu vytvoření a transformace v reálné prvky uživatelského rozhraní.
    Tato část práce vyvrcholila výběrem vhodného nástroje pro návrh \ac{ui}, kterým se stal Figma a následným vytvořením návrhu webové aplikace pro prodej vstupenek.

    Poslední část práce se zaměřila na samotný vývoj webové aplikace pro prodej vstupenek s využitím nejmodernějších technologií a nástrojů pro vývoj frontendu, jako jsou mimo jiné React, TypeScript, Next.js a TailwindCSS.\
    Tato část se převážně týkala převedení návrhové předlohy do reálného kódu, včetně všech interaktivních prvků, řešení a překonání všech výzev, které se vyskytly během vývoje a zajištění připravenosti aplikace k nasazení.

    Celkově bylo cílem vytvořit jedinečný systém prodeje vstupenek s rezervací míst, který není jen funkční, ale také uživatelsky přívětivý a intuitivní.
    Taktéž bylo ukázáno, že dobře promyšlený návrh, pečlivé plánování a správné použití technologií může vést k vytvoření aplikace, která překoná očekávání uživatelů a poskytne jim vynikající uživatelský zážitek.
\end{section}

%%% Sekce – Zhodnocení výstupu
%%%%%%% Wording: ✅
%%%%%%% Styling: ✅
%%%%%%% References: ✅
%%%%% Grammar: ✅
%%% --------------------------------------------------------------
\begin{section}{Zhodnocení výstupu}
    \label{sec:zaver-zhodnoceni}
    Zhodnocení finálního výstupu práce v kontextu jejích cílů a úkolů poskytuje hmatatelnou míru úspěchu.
    Cílem bylo vytvořit efektivní a uživatelsky přívětivý systém pro prodej vstupenek s rezervací míst.

    Výstup této práce je vyhodnocen z perspektivy sebereflexe, zaměřující se na pohodlí, estetiku, responsivitu a intuitivnost uživatelského rozhraní.

    Z hlediska pohodlí systém vyniká.
    Proces výběru sedadel, jejich přidání do nákupního košíku a dokončení objednávky je zjednodušený a lze jej snadno pochopit.
    Zjednodušuje cestu uživatele od výběru vstupenky k dokončení nákupu, což byl jeden z hlavních cílů této práce.

    Z hlediska estetiky je design systému vizuálně příjemný a čistý.
    Barevné kódování sedadel, jasný rozdíl mezi jednotlivými sekcemi a plynulé přechody přispívají k uživatelskému rozhraní, které je nejen funkční, ale také vizuálně přitažlivé.

    Responsivita byla dalším klíčovým aspektem, který byl při návrhu a vývoji této aplikace zohledněn.
    Systém je navržen tak, aby se přizpůsobil různým velikostem obrazovky, což umožňuje přístup k aplikaci z různých zařízení bez ztráty funkcionality.
    Interaktivní mapa, nákupní košík a dokončení objednávky vykazují dobrou responsivitu, na různých zařízeních, což jednoznačně přispívá k celkové uživatelské zkušenosti.

    Intuitivnost systému byla během návrhu a vývoje hlavním zaměřením.
    Cílem bylo vytvořit rozhraní, kterým uživatelé mohou navigovat bez většího úsilí.
    Od interaktivní mapy sedadel přes nákupní košík až po dokončení objednávky je tok akcí navržen tak, aby byl intuitivní, přičemž každý krok logicky vede k dalšímu.

    Implementace aplikace, která zahrnovala použití moderních frontendových technologií, vedla k systému, který je robustní, efektivní a schopný zpracovávat velké objemy uživatelských interakcí.
    Kód je dobře strukturovaný a čistý, což naznačuje, že během práce byly dodržovány dobré praktiky softwarového vývoje.

    V závěru lze výstup považovat za velmi úspěšný.
    Splňuje počáteční cíle a poskytuje efektivní, robustní a uživatelsky přívětivý systém pro prodej vstupenek.
    Návrhová a implementační fáze vedly k aplikaci, která dobře funguje a splňuje potřeby koncových uživatelů.

    Projekt však nedokáže poskytnout kompletní řešení pro prodej vstupenek od začátku do konce kvůli nedostatku plně implementovaného backendu.
    Zatímco frontend nabízí plynulou uživatelskou zkušenost, absence komplexního backendového systému zanechává prostor pro zlepšení a rozšíření v budoucnu.
\end{section}

%%% Sekce – Možná budoucí zlepšení
%%%%%%% Wording: ✅
%%%%%%% Styling: ✅
%%%%%%% References: ✅
%%%%% Grammar: ✅
%%% --------------------------------------------------------------
\begin{section}{Možná budoucí zlepšení}
    \label{sec:zaver-budoucnost}
    Navzdory zdárnému úspěchu práce je vždy prostor pro vylepšení a přidání nových funkcí.
    Následující seznam obsahuje některé oblasti z možných vylepšení, které by mohly být implementovány v budoucích iteracích.

    \textbf{Úplný rozsah aplikace}: V současné době je navržená aplikace zaměřena na jedno místo pořádající jednu událost.
    Tento rozsah, ačkoli je v současné podobě užitečný, je přiznáváno, že je v širším měřítku robustního, reálného systému pro prodej vstupenek omezený.
    Rozšíření aplikace tak, aby zahrnovala více míst a mnoho událostí, je značným dalším krokem.
    Toto rozšíření by neznamenalo pouze zvětšení stávajícího systému, ale vyžadovalo by strategické plánování pro vytvoření sofistikovanější navigační struktury.
    Tato struktura by zajistila bezproblémové přepínání mezi různými místy a událostmi, přičemž by zachovala intuitivní rozhraní, které je středem uživatelského zážitku.

    \textbf{Implementace backendu}: Současná iterace je aplikace zaměřená na frontend, která, ačkoli je funkční a demonstruje zaměřené koncepty, chybí plně integrovaný backendový systém.
    Přidání tohoto systému, který zahrnuje API a dedikovanou databázi, by prototyp přeměnilo v komplexní full-stack aplikaci.
    Backend by umožnil manipulaci a správu skutečných dat, včetně údajů o uživatelích, objednávek a plateb.
    Implementace této složité vrstvy by aplikaci posunula na vyšší úroveň funkčnosti a poskytla uživateli realističtější a robustnější zážitek.

    \textbf{Integrace platební brány}: Jedním z hlavních vylepšení na obzoru je integrace skutečné platební brány zajišťující bezpečné a spolehlivé zpracování plateb.
    V současné době aplikace uživatele vede procesem až do okamžiku nákupu, který záměrně přeskočí z důvodu absence platebního mechanismu.
    Integrace platební brány by umožnila uživatelům dokončit objednávku a získat skutečné vstupenky.

    \textbf{Aktualizace v reálném čase}: V prostředí, kde se dostupnost míst může změnit během několika sekund, jsou aktualizace v reálném čase nezbytné.
    Integrace takové funkce, například prostřednictvím technologie \ac{ws}, by mohla výrazně zlepšit celkový uživatelský zážitek.
    Tyto aktualizace by mohly zahrnovat změny v dostupnosti míst, změny v cenách a změny v celkovém počtu dostupných vstupenek.

    \textbf{Lokalizace a přístupnost}: Pro rozšíření dosahu a použitelnosti aplikace by mohla být provedena vylepšení v oblasti lokalizace (\foreign{i18n}\footnote{\foreign{i18n} je zkratka pro \foreign{internationalization} (mezinárodní) a znamená proces přizpůsobení aplikace pro použití v různých jazycích a regionech.}) a přístupnosti (\foreign{a11y}\footnote{\foreign{a11y} je zkratka pro \foreign{accessibility} (přístupnost) a znamená proces přizpůsobení aplikace pro použití uživateli s různými schopnostmi.}).
    Lokalizace by zahrnovala implementaci podpory vícejazyčnosti, čímž by se aplikace stala přístupnou pro širší publikum.
    Paralelně se zaměřením na přístupnost by se zajistilo, že aplikace bude přístupná uživatelům s různými schopnostmi a poskytovala by stejně příjemný zážitek pro všechny.

    \textbf{Backoffice řešení}: Nakonec by vývoj komplexního backoffice řešení velmi usnadnil správu aplikace.
    Takovéto řešení by umožnilo efektivní kontrolu různých aspektů aplikace, včetně správy událostí, přidělování sedadel a sledování objednávek.
    Kromě toho by vytvoření editoru mapy sedadel nabídlo možnost snadno upravovat a přizpůsobovat rozvržení sedadel pro různá místa a akce, poskytující vysoce flexibilní nástroj pro administraci.

    Tato výše zmíněná potenciální vylepšení a rozšíření slouží jako mapa pro přeměnu této počáteční iterace v komplexní a plnohodnotnou aplikaci.
    Každé vylepšení by přispělo k tomu, aby aplikace byla robustnější, univerzálnější a uživatelsky přívětivější, čímž by se aplikace dostala na úroveň profesionálního systému pro prodej vstupenek.
\end{section}

%%% Sekce – Splnění cílů práce
%%%%%%% Wording: ✅
%%%%%%% Styling: ✅
%%%%%%% References: ✅
%%%%% Grammar: ✅
%%% --------------------------------------------------------------
\begin{section}{Splnění cílů práce}
    \label{sec:zaver-cile}
    V úvodu práce byly definovány 4 cíle.

    Prvním cílem byla identifikace klíčových prvků a funkčností systému pro prodej vstupenek s rezervací míst.
    Tyto prvky byly identifikovány a popsány v kapitole~\fullref{ch:identifikace} a jmenovitě šlo o interaktivní mapu sedadel, nákupní košík a proces dokončení objednávky.

    Druhý cíl se zaměřoval na navržení uživatelského rozhraní na základě definice uživatelských příběhů.
    Ty byly v rámci kapitoly~\fullref{ch:navrh-uzivatelskeho-rozhrani} sestaveny v sekci~\ref{subsec:navrh-ui-uzivatelske-pribehy-konkretni} a následně byly využity při návrhu uživatelského rozhraní v sekci~\ref{sec:navrh-ui-navrh-produktu}.
    Výsledkem je návrh uživatelského rozhraní skrze nástroj Figma, který je k práci přiložen v příloze~\fullref{appendix:ui-design}.

    Třetí cíl se týkal vývoje responsivní webové aplikace pro prodej vstupenek s rezervací míst.
    Tento cíl byl splněn v rámci kapitoly~\fullref{ch:implementace}, která detailně popisuje vývoj aplikace i její architekturu.
    Zdrojový kód aplikace je k práci přiložen v příloze~\fullref{appendix:source-code}.

    Poslední, čtvrtý cíl, se zaměřoval na nasazení aplikace do produkčního prostředí.
    Toho bylo docíleno v rámci kapitoly~\ref{ch:implementace}, kde byl v sekci~\ref{sec:implementace-deployment} popsán proces nasazení aplikace na platformu Vercel.

    Všechny tyto cíle byly splněny a výsledkem je funkční webová aplikace pro prodej vstupenek s rezervací míst, která je dostupná na adrese \url{https://seating-map.vercel.app}.
\end{section}

%%% Sekce – Finální myšlenky
%%%%%%% Wording: ✅
%%%%%%% Styling: ✅
%%%%%%% References: ✅
%%%%% Grammar: ✅
%%% --------------------------------------------------------------
\begin{section}{Finální myšlenky}
    \label{sec:zaver-myslenky}
    Závěrem je zřejmé, že tato práce byla pozoruhodnou cestou, která zkoumala různé aspekty vývoje webových aplikací, od složitého návrhu uživatelského rozhraní po komplexní techniky frontendového vývoje.

    Proces vytváření komplexního a funkčního systému pro prodej vstupenek s intuitivním uživatelským rozhraním se ukázal jako cenný cvičný úkol v aplikaci teoretických konceptů na praktická řešení.
    Zároveň zdůraznil důležitost uživatelsky zaměřeného návrhu a roli nových technologií při zjednodušování složitých úkolů.

    Práce navíc zdůraznila význam správného plánování a organizace při řízení rozsáhlých vývojových úkolů.
    Ukázala, že architektura projektu a správná volba technologií jsou klíčové faktory, které mohou výrazně ovlivnit rychlost a efektivitu vývoje, stejně jako celkovou kvalitu výsledného produktu.

    V závěru lze říci, že tato práce shrnuje složitý proces tvorby webové aplikace zaměřené na uživatele a podtrhuje hodnotu uživatelského zážitku při vývoji frontendu a návrhu uživatelského rozhraní.
    Aplikace položila pevné základy pro budoucí kompletní řešení pro prodej vstupenek s rezervací míst – její potenciál pro růst a vylepšení představuje povzbudivý začátek směrem k vytváření intuitivnějších a efektivnějších aplikací.
\end{section}


