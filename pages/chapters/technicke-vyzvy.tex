%%%%% Kapitola 2 - Technické výzvy
%%%%% ------------------------------------------------------------
\chapter{Technické výzvy}
TODO: technické výzvy při vývoji systému online prodeje vstupenek s rezervací míst

%%% Sekce - Webová stránka akce
%%% --------------------------------------------------------------
\section{Webová stránka akce}
TODO: webová stránk akce, důležitost SEO, jakým způsobem se to dělá, sdílení na sociálních sítích, atd.

%%% Sekce - Real-time dostupnost dat
%%% --------------------------------------------------------------
\section{Real-time dostupnost dat}
TODO: důležitost real-time dostupnosti dat, jakým způsobem se to dělá, atd.

%%% Sekce - Seating mapa
%%% --------------------------------------------------------------
\section{Seating mapa}
TODO: co je seating mapa, jaké mohou být typy seating map, jakým způsobem se vytvářejí, atd.

%%% Sekce - Administrace a backoffice
%%% --------------------------------------------------------------
\section{Administrace a backoffice}
TODO: popis administrace a backoffice, co je to, proč je potřeba, jak funguje, atd.

%%% Sekce - Rozhraní pro pořadatele
%%% --------------------------------------------------------------
\section{Rozhraní pro pořadatele}
TODO: dashboard rozhraní pro pořadatele, statistiky, analýza, exporty dat, atd.

%%% Sekce - Rezervace míst
%%% --------------------------------------------------------------
\section{Rezervace míst}
TODO: proč je potřeba rezervovat místa, čemu předchází, jakým způsobem se to dělá, atd.

%%% Sekce - Platební poskytovatelé
%%% --------------------------------------------------------------
\section{Platební poskytovatelé}
TODO: možnosti platebních metod při online prodeji vstupenek, integrace s poskytovateli\\
TODO: platební brány, popis, co znamená být platební bránou, výhody a nevýhody\\
TODO: Apple Pay, Google Pay, jak fungují, výhody a nevýhody
