%%% Sekce – Struktura a architektura projektu
%%%%% Wording: ✅
%%%%% Styling: ✅
%%%%% References: ✅
%%% --------------------------------------------------------------
\section{Struktura a architektura projektu}
\label{sec:implementace-architektura}
Zahájení a organizace nového projektu je klíčovou fází, která vytváří základy pro celou aplikaci.
Tato část poskytuje přehled procesu a úvah zahrnutých do počátečního nastavení projektu a definování jeho architektury.
Cílem bylo vyvinout frontendovou aplikaci, která bude udržitelná, škálovatelná a dobře strukturovaná, dodržující nejlepší postupy současného vývoje frontendu.

%%% Podesekce – Vytvoření projektu
%%%%% Wording: ✅
%%%%% Styling: ✅
%%%%% References: ✅
%%% --------------------------------------------------------------
\subsection{Vytvoření projektu}
\label{subsec:implementace-architektura-vytvoreni-projektu}
Projekt byl vytvoření pomocí příkazu \mintinline{bash}{npx create-next-app --typescript}, nástroje pro rychlé nastavení nového projektu, který poskytuje Next.js.
Příznak \mintinline{bash}{--typescript} byl zahrnut, aby bylo jasné, že se má použít TypeScript místo samotného JavaScriptu.
Tento příkaz vytvoří nový adresář se základní strukturou projektu Next.js a sadou přednastavených souborů\cite{n_nextjs_org_docs}.

Jakmile bylo počáteční nastavení projektu dokončeno, následovalo odstranění vygenerovaného boilerplate kódu a přidání vybraných knihoven a technologií pro projekt.
Tento proces zahrnoval odstranění výchozích \ac{css} souborů, jelikož projekt využívá Tailwind CSS pro stylování a přidání MantineUI, komplexní knihovny pro uživatelské rozhraní Reactu.
Přidání ostatních technologií a knihoven zmíněných v sekci~\ref{sec:implementace-technologie}, jako jsou Prettier, ESLint, axios a react-query, byly také součástí této počáteční fáze nastavení.

V tomto projektu bylo pro správu balíčků upřednostněno použití \mintinline{bash}{pnpm}.
Vyniká svým efektivním zacházením s Node.js moduly, čímž zajišťuje rychlejší a spolehlivější \foreign{build proces}\footnote{\foreign{Build proces} je proces, kdy se zdrojový kód překládá, neboli transpiluje, do spustitelného formátu, který může být spuštěn na cílovém zařízení.}\cite{p__pnpm_io}.

Dalším krokem bylo vytvoření celkové struktury projektu, která vyhovuje potřebám aplikace a podporuje osvědčené postupy.
Dobře promyšlená a organizovaná struktura projektu je zásadní pro udržení čitelnosti kódu, škálovatelnosti a snadné navigace, zejména když \foreign{codebase} roste.

%%% Podesekce – Základní architektura
%%%%% Wording: ✅
%%%%% Styling: ✅
%%%%% References: ✅
%%% --------------------------------------------------------------
\subsection{Základní architektura}
\label{subsec:implementace-architektura-zakladni}
Efektivně definovaná struktura projektu hraje klíčovou roli při zajišťování udržovatelnosti, škálovatelnosti a čitelnosti kódu.
Logická, jasná a konzistentně udržovaná struktura umožňuje vývojářům snadno procházet a porozumět aplikaci, i když se \foreign{codebase} časem rozrůstá a vyvíjí.

Tento projekt využívá modulární architekturu, kde je kód kategoricky organizován do odlišných modulů na základě jejich specifických funkcí.
Tato metodologie slouží ke zlepšení čitelnosti a udržovatelnosti kódové základny zapouzdřením souvisejících logických komponent do samostatných modulů\cite{p_article_react_folder_structure}.

Níže je podrobnější pohled na složky a soubory hlavní struktury projektu a jejich účely:

\begin{lstlisting}[language={[LaTeX]TeX},caption={Struktura projektu},label={lst:project-structure}]
src/
|-- lib/
|   |-- components/
|   |-- features/
|   |-- graphics/
|   |-- hooks/
|   |-- types/
|   `-- utils/
|-- pages/
`-- styles/
next.config.js
package.json
.eslintrc.js
prettier.config.js
tsconfig.json
tailwind.config.js
\end{lstlisting}

\begin{itemize}
    \item \textbf{src:} Toto je hlavní adresář obsahující veškerý zdrojový kód aplikace.
    \item \textbf{src/lib:} Tento adresář obsahuje hlavní funkce projektu, které jsou dále rozděleny do dalších podadresářů.
    \item \textbf{src/lib/components:} Tento adresář obsahuje znovupoužitelné komponenty, které jsou strukturovány do vlastních podadresářů, které obsahují TypeScriptové soubory pro jejich otypování a jejich implementaci spolu s \foreign{barrel}\footnote{\foreign{Barrel} soubor je jednoduše soubor pojmenovaný jako \foreign{index}, který importuje ze všech souborů v rámci adresáře a znovu je exportuje z jednoho místa.} souborem pro snadnější importování.
    \item \textbf{src/lib/features:} Tento adresář se skládá z konkrétních funkcí nebo logických částí aplikace, jako je \ac{api} klient, správa košíku či mapa sedadel.
    \item \textbf{src/lib/graphics:} Tento adresář uchovává grafické zdroje, jako jsou \ac{svg} ikony používané v aplikaci.
    \item \textbf{src/lib/hooks:} Tato složka obsahuje vlastní pomocné React hooky.
    \item \textbf{src/lib/types:} Tento adresář uchovává běžné typy TypeScript používané v celé aplikaci.
    \item \textbf{src/lib/utils:} Tento adresář obsahuje různé pomocné funkce používané v aplikaci.
    \item \textbf{src/pages:} Obsahuje komponenty stránek a \ac{API} endpointy.
    \item \textbf{src/styles:} Obsahuje globální styly aplikace.
    \item \textbf{next.config.js:} Konfigurační soubor pro Next.js, který umožňuje přizpůsobení různých aspektů frameworku.
    \item \textbf{package.json:} Soubor obsahuje metadata a seznam použitých knihoven v projektu.
    \item \textbf{.eslintrc.js, prettier.config.js, tsconfig.json:} Jsou konfigurační soubory pro ESLint, Prettier a TypeScript.
    \item \textbf{tailwind.config.js:} Je konfigurační soubor pro Tailwind CSS\@.
\end{itemize}

Každá z těchto složek a souborů hraje v aplikaci klíčovou roli a přispívá ke struktuře a architektuře projektu.
V dalších sekcích budou podrobněji popsány konkrétní implementační detaily projektu, počínaje interaktivní mapou sedadel.
