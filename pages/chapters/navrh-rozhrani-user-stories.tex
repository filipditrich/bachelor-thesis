%%% Sekce - Uživatelské příběhy
%%%%% Wording: ✅
%%%%% Styling: ✅
%%%%% References: ✅
%%%%% Grammar: ✅
%%% --------------------------------------------------------------
\section{Uživatelské příběhy}
\label{sec:navrh-uzivatelskeho-rozhrani-uzivatelske-pribehy}
Při navrhování uživatelského rozhraní nejde pouze o estetiku nebo funkčnost; je vyžadováno pochopení potřeb a očekávání uživatele.
Zahrnuje vytváření cesty, která uživatele bezproblémově provede aplikací a zároveň zajistí, aby mohli své úkony vykonávat efektivně a s potěšením.
Technika, která se často používá v návrhu \ac{ui}/\ac{ux} k dosažení tohoto cíle, se nazývá \textit{User Stories} (uživatelské příběhy).

Uživatelské příběhy jsou stručné, přímočaré popisy funkce nebo funkcionality, vyprávěné z pohledu uživatele.
Slouží jako nástroj, který pomáhá udržovat návrh zaměřený na uživatele a zajišťuje, že konečný produkt efektivně splňuje jeho potřeby.
Tyto příběhy kladou důraz na to, čeho uživatelé chtějí dosáhnout, podporují empatii a podporují návrhový proces, který se zaměřuje na uživatele\cite{w_articles_user_stories_a_foundation_for_ui_design}.
Porozumění roli uživatelských příběhů při návrhu \ac{ui} webového řešení pro prodej vstupenek je klíčové pro efektivní splnění potřeb koncových uživatelů.

%%% Podsekce - Co jsou User Stories
%%%%% Wording: ✅
%%%%% Styling: ✅
%%%%% References: ✅
%%%%% Grammar: ✅
%%% --------------------------------------------------------------
\begin{subsection}{Co jsou uživatelské příběhy}
    \label{subsec:navrh-ui-uzivatelske-pribehy-co-jsou}
    Uživatelské příběhy jsou součástí agilních vývojových praktik, široce používaných v návrhu \ac{ui}/\ac{ux} k zachycení zjednodušených popisů potenciálních funkcí aplikace z pohledu koncových uživatelů.
    Slouží jako rychlý a jednoduchý způsob, jak popsat uživatele, co chtějí a proč to chtějí\cite{w_articles_user_stories_a_foundation_for_ui_design}.
    Každá \foreign{User Story} následuje strukturovaný formát:

    \begin{gray-box}{Formát uživatelského příběhu}
        ``\textbf{Jako} [\textit{typ uživatele}] \textbf{chci} [\textit{vykonat nějakou akci}], \textbf{abych} [\textit{dosáhl nějakého cíle}].``
    \end{gray-box}

    \pagebreak
    V tomto formátu:
    \begin{itemize}
        \item \textbf{Typ uživatele} pomáhá definovat roli uživatele, který bude používat danou funkcionalitu.
        \item \textbf{Vykonat nějakou akci} umožňuje zjistit, co chce uživatel pomocí dané funkcionality udělat nebo čeho chce dosáhnout.
        \item \textbf{Dosáhl nějakého cíle} vysvětluje základní motivaci nebo hodnotu, kterou uživatel získá provedením akce.
    \end{itemize}

    Tento formát je velmi užitečný při vytváření uživatelských příběhů, jelikož pomáhá udržovat stručnost, jednoznačnost a zároveň poskytuje dostatek informací, aby bylo možné pochopit, co uživatel chce a proč to chce.

    Uživatelské příběhy hrají také klíčovou roli při definování akceptačních kritérií, která dále podrobně popisují, jak by měla určitá funkce fungovat z pohledu uživatele.
    To pomáhá stanovit jasnou představu o účelu a očekávaném chování funkce, čímž usměrňuje její vývoj a testování\cite{w_articles_user_stories_a_foundation_for_ui_design}.

    V kontextu navrhovaného uživatelského rozhraní pro webové řešení prodeje vstupenek mohou tyto uživatelské příběhy pomoci přesně určit funkcionality, které jsou pro uživatele nejdůležitější.
    Pomáhají porozumět potencionálním uživatelům –~návštěvníkům událostí, jejich potřebám (jako jsou např.\ prohlížení místa konání, výběr sedadel), jejich akce (přidání vstupenek do košíku, přechod k zaplacení) a jejich motivaci (užít si bezproblémový nákup vstupenek).

    Následující sekce se zabývá tím, jak lze uživatelské příběhy konkrétně použít k navrhování uživatelského rozhraní.
\end{subsection}

%%% Podsekce - Psaní efektivních uživatelských příběhů
%%%%% Wording: ✅
%%%%% Styling: ✅
%%%%% References: ✅
%%%%% Grammar: ✅
%%% --------------------------------------------------------------
\begin{subsection}{Psaní efektivních uživatelských příběhů}
    \label{subsec:navrh-ui-uzivatelske-pribehy-psani-efektivnich}
    Při psaní efektivních uživatelských příběhů aplikace je klíčové porozumět perspektivě koncového uživatele.
    Tento proces vyžaduje identifikaci potřeb, motivací a požadovaných výsledků uživatele při používání aplikace.

    První krok je identifikace a pochopení různých \foreign{personas} neboli \textbf{typů uživatelů}, kteří budou pravděpodobně s aplikací interagovat.
    V případě zkoumané aplikace je primárním uživatelem někdo, kdo má zájem o nákup vstupenek na události.
    Sekundární uživatelé, jako jsou organizátoři akcí nebo manažeři prostorů, však mohou také s aplikací interagovat s odlišnými požadavky na funkcionalitu, nicméně jejich potřeby nejsou v rámci této práce důležité.

    Dalším krokem je zjištění, čeho uživatelé \textbf{chtějí dosáhnout}.
    To může zahrnovat jednoduché úkony, jako například \textit{prohlížení mapy místa konání}, nebo složitější, jako \textit{rezervace konkrétních sedadel}.
    Každý uživatelský příběh by měl zůstat stručný a zaměřený na jednu akci.

    Posledním krokem je definování \textbf{požadovaného výsledku}, který uživatel získá provedením daného úkonu neboli definování motivace uživatele.
    Tento krok je klíčový, jelikož dále napomáhá při prioritizaci funkcí na základě získané hodnoty, kterou poskytuje uživateli.

    Při tvoření uživatelských příběhů je užitečné dodržovat princip \foreign{INVEST} (\foreign{Independent}, \foreign{Negotiable}, \foreign{Valuable}, \foreign{Estimable}, \foreign{Small}, \foreign{Testable}).
    Tento princip zajišťuje, že každý uživatelský příběh je dobře definován a má potřebné charakteristiky pro efektivní implementaci v procesu vývoje\cite{w_glossary_invest}.

    Příkladem takového uživatelského příběhu v rámci zkoumaného webového řešení prodeje vstupenek s rezervací míst může být následující:

    \begin{gray-box}{Ukázka uživatelského příběhu}
        \textit{``Jako zákazník si chci být schopen vybrat konkrétní sedadlo, abych si mohl zakoupit vstupenku na akci.``}
    \end{gray-box}

    Tento uživatelský příběh je nezávislý na ostatních uživatelských příbězích, je jednoduchý a snadno pochopitelný, poskytuje hodnotu uživateli a je snadno testovatelný.
    Při dodržení tohoto principu mohou uživatelské příběhy poskytnout cenný náhled do toho, jak by měla výsledná aplikace fungovat z pohledu uživatele.
\end{subsection}

%%% user story 1
\newcommand{\userstoryvenuemap}{
    \userstory{1}{Vizualizace místa konání}{Jako zákazník, chci vidět, jak vypadá místo konání, abych si mohl vybrat místo, které mi bude vyhovovat.}
}
%%% user story 2
\newcommand{\userstoryseatselection}{
    \userstory{2}{Výběr sedadla}{Jako zákazník, si chci označit či odznačit konkrétní sedadla, abych si mohl vybrat místa, která mi budou vyhovovat.}
}
%%% user story 3
\newcommand{\userstoryshoppingcart}{
    \userstory{3}{Nákupní košík}{Jako zákazník, chci mít jasný přehled o přidaných vstupenkách do nákupního košíku, abych měl přehled o svém nákupu.}
}
%%% user story 4
\newcommand{\userstorycheckout}{
    \userstory{4}{Vyřízení objednávky}{Jako zákazník, chci jasný a jednoduchý proces vyřízení objednávky, abych mohl svůj nákup vstupenek snadno dokončit.}
}

%%% Podsekce - Návrh konktrétních uživatelských příběhů
%%%%% Wording: ✅
%%%%% Styling: ✅
%%%%% References: ✅
%%%%% Grammar: ✅
%%% --------------------------------------------------------------
\begin{subsection}{Návrh konkrétních uživatelských příběhů}
    \label{subsec:navrh-ui-uzivatelske-pribehy-konkretni}
    Na základě pochopení získaného v předchozích sekcích se lze nyní zaměřit na konstrukci konkrétních uživatelských příběhů pro zkoumanou aplikaci zaměřenou na prodej vstupenek s rezervací míst.
    Nejprve je však nutné definovat hlavní typ uživatele, který bude tuto aplikaci používat.

    V základu lze říci, že hlavní rolí uživatele je potencionální zákazník, který má zájem o nákup vstupenky na konkrétní událost.
    Pro další účely bude použito pouze záměrné označení \textbf{zákazník}.
    Každý příběh bude představen ve stanoveném formátu a na závěr bude diskutováno o tom, jak daný příběh ovlivňuje návrh uživatelského rozhraní.

    \userstoryvenuemap

    Tento uživatelský příběh zdůrazňuje důležitost jasné a intuitivní vizualizace místa konání.
    Mapa musí poskytovat přesnou reprezentaci uspořádání sedadel a nabízet dostatek detailů, aby uživatelé mohli snadno vybrat místo, které jim vyhovuje.

    \userstoryseatselection

    Flexibilní výběr sedadla je klíčovým aspektem pro uživatele, jelikož umožňuje volnost ve výběru sedadel.
    Uživatelé by měli mít možnost vybrat si konkrétní místo, které jim vyhovuje, a měli by mít možnost si vybrat případně i více míst, pokud si přejí sedět například s přáteli nebo rodinou.

    Uživatelské rozhraní by tedy mělo uživatelům umožnit snadno vybrat a zrušit výběr míst, aby mohli vyzkoušet různé možnosti výběru, které jsou k dispozici.

    \userstoryshoppingcart

    Tento uživatelský příběh zdůrazňuje důležitost uživatelského rozhraní nákupního košíku se vstupenkami.
    Uživatelé by měli mít možnost snadno zobrazit, jaké vstupenky mají v nákupním košíku, a měli by mít možnost snadno upravovat jeho obsah.

    To vyžaduje jednoduché a přístupné uživatelské rozhraní spravující komplexní funkci nákupního košíku.

    \userstorycheckout

    Poslední zmíněný uživatelský příběh se zaměřuje na proces vyřízení objednávky.
    Zdůrazňuje potřebu jednoduchého a intuitivního uživatelského rozhraní, které umožní uživatelům snadno dokončit svůj nákup vstupenek.

    Uživatelské rozhraní dokončení objednávky by tedy mělo minimalizovat komplexitu a zmatečnost, čímž zajistí uživatelům důvěru při dokončování své objednávky.

    Z důvodu zachování přehlednosti a jednoduchosti, ačkoliv by mohlo být vytvořeno více uživatelských příběhů, budou v této práci použity pouze tyto čtyři hlavní uživatelské příběhy.
    Tyto příběhy, budou v dále použity jako základní stavební kameny pro návrh uživatelského rozhraní a jeho komponent v rámci vyvíjené webové aplikace.

    Pro to bude ale nejdříve nutné vybrat vhodný návrhový nástroj, který bude schopen tyto příběhy transformovat do funkčního designu.
    Další sekce se nyní bude zabývat nástroji dostupnými pro návrh \ac{ui} a bude diskutováno o důvodech výběru konkrétního nástroje pro tento projekt.
\end{subsection}
