%%% Kapitola – Návrh uživatelského rozhraní
%%%%% Wording: ✅
%%%%% Styling: ✅
%%%%% References: ✅
%%%%% Grammar: ✅
%%% --------------------------------------------------------------
\chapter{Návrh uživatelského rozhraní}
\label{ch:navrh-uzivatelskeho-rozhrani}
Ve světě digitálních produktů a jejich designu jsou uživatelské rozhraní, z anglického \foreign{\acf{ui}}, a uživatelský zážitek, z anglického \foreign{\acf{ux}}, dva pojmy, které se často zaměňují, ačkoli se jedná o velmi odlišné aspekty procesu vývoje produktu\cite{c_ux_design_the_difference_between_ux_and_ui_design_a_laymans_guide}.
Tato kapitola si klade za cíl představit koncepty \ac{ui} a \ac{ux}, prozkoumat jejich vzájemný vztah a zabývat se specifiky návrhu \ac{ui} pro zkoumanou a vyvíjenou aplikaci prodeje vstupenek s rezervací míst.

\textbf{\ac{ui}} se vztahuje k vizuálním prvkům produktu, se kterými uživatel interaguje – tedy tlačítkům, textu, ikonografii, formulářům a všem vizuálním prvkům, které umožňují uživateli interagovat s produktem\cite{c_ux_design_the_difference_between_ux_and_ui_design_a_laymans_guide}.
V kontextu aplikace pro prodej vstupenek s rezervací míst se \ac{ui} vztahuje například k interaktivnímu plánu sedaček, výběru vstupenek, tlačítku pro přechod k dokončení objednávky nebo nákupnímu košíku.

\textbf{\ac{ux}} je na druhou stranu celkový zážitek uživatele při interakci s produktem.
Je ovlivněn snadností použití, hodnotou, kterou uživatel z produktu získává, a emocemi, které jsou při interakci vyvolány.
\ac{ux} bere v potaz celou cestu uživatele, od okamžiku, kdy uživatel do aplikace vstoupí, až po okamžik, kdy dokončí nákup\cite{c_ux_design_the_difference_between_ux_and_ui_design_a_laymans_guide}.

Souhra \ac{ui} a \ac{ux} je v procesu návrhu produktu klíčová.
Dobře navržené \ac{ui} usnadňuje \ac{ux}.
Například intuitivně navržený plán sedadel (\ac{ui}) může proces výběru sedadla zpříjemnit a zjednodušit (\ac{ux}).

Následující sekce této kapitoly prozkoumají základní principy návrhu \ac{ui} a možná použití Maslowovy hierarchie, za účelem návrhu rozhraní více zaměřeného na uživatele.
Dále budou uvedeny a porovnány různé nástroje, které jsou k dispozici pro návrh \ac{ui} a důvody rozhodnutí pro konkrétní nástroj.
Následně budou analyzovány poznatky z kapitoly~\ref{ch:identifikace} z hlediska \ac{ui}/\ac{ux} se zaměřením na tzv.\ uživatelské příběhy, které tvoří základ \ac{ux} designu.

Závěr této kapitoly bude věnován návrhu uživatelského rozhraní z předem definovaných uživatelských příběhů a návrhu interaktivního prototypu aplikace.
