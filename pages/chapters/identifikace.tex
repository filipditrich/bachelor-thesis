%%%%% Kapitola - Identifikace klíčových částí
%%%%% Wording: ✅
%%%%% Styling: ✅
%%%%% References: ✅
%%%%% ------------------------------------------------------------
\chapter{Identifikace klíčových částí}
\label{ch:identifikace}
Vývoj efektivního a intuitivního webového řešení systému pro prodej vstupenek s rezervací míst je komplexní úkol, který vyžaduje důkladné porozumění a pečlivou organizaci klíčových prvků, které tvoří základ takových aplikací.
Tato kapitola se proto snaží provést podrobnou analýzu a specifikaci těchto prvků, z nichž každý hraje důležitou roli ve funkcionalitě a spokojenosti koncových uživatelů aplikace.
Tyto prvky budou zkoumány na základě reálných existujících systémů určených k prodeji vstupenek s podporou rezervace míst, které jsou v současné době používány v rámci České republiky - konkrétně se jedná o systémy Ticketportal\footnote{\url{https://www.ticketportal.cz/}}, Ticketmaster\footnote{\url{https://www.ticketmaster.cz/}}, GoOut\footnote{\url{https://goout.net/}} a NFCtron~Tickets\footnote{\url{https://www.nfctron.cz/}}.

Primární záměr této kapitoly je rozbor a popis interaktivní mapy sedadel, která je středobodem takového systému, jelikož umožňuje uživatelům vizualizovat místo konání a vybrat si svá požadovaná místa.
Kapitola poskytuje komplexní vysvětlení různých prvků souvisejících s touto funkcí, jako je uspořádání a struktura mapy, použití barevného kódování k rozlišení sedadel, ovládací prvky pro interakci s mapou, zobrazení informací o sedadlech a dostupnost dat.

Následující sekce se zaměřuje na mechanismus nákupního košíku, který je další esensiální součástí online \foreign{e-commerce}\footnote{\foreign{E-commerce} je zkratka pro \enterm{electronic commerce}, tedy elektronický obchod. Jedná se o obchodní transakce prováděné prostřednictvím internetu.} systémů.
Konkrétně tato sekce zkoumá specifické požadavky nákupního košíku v online systému pro prodej vstupenek.
Zabývá se primárně správou dat týkajících se vybraných sedadel a souvisejících informací o vstupenkách a procesem rezervace sedadel, kdy jsou místa dočasně držena pro uživatele, dokud není nákup dokončen.
Sekce také řeší aspekt uživatelského rozhraní, zkoumající, jak je nákupní košík uživatelům prezentován, aby usnadnil pochopení a interakci s ním.

Poslední sekce této kapitoly, sekce~\ref{sec:identifikace-dokonceni-objednavky}, se zabývá analýzou procesu objednávky.
Tato část systému zahrnuje konečné dokončení nákupu, které zahrnuje odeslání osobních údajů, výběr vhodného způsobu platby, kontrolu souhrnu objednávky a konečné potvrzení objednávky.
Význam každé fáze tohoto postupu by neměl být podceňován, protože společně přispívají k efektivitě systému a vyžadují pečlivé zvážení, aby byla zaručena bezpečnost, soukromí a uživatelsky přívětivý zážitek.

Zkoumáním těchto částí vytváří tato kapitola základ pro pozdější fáze návrhu a implementace konkrétního prototypu aplikace.
Důkladné porozumění těmto komponentám povede k vývoji uživatelského rozhraní, které je nejen snadno použitelné, ale také vysoce efektivní.
Podobně poznatky získané z tohoto zkoumání ovlivní výběr technologií, návrh architektury a přístupy k vývoji během procesu implementace.
