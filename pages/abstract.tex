%%%
%%%  ABSTRAKT
%%%
%%%  * soubor obsahující abstrakt bakalářské práce
%%%
%%%  ===========================================================================
\newpage
\pagestyle{plain}

%%% abstrakt česky
\vbox to 0.5\vsize{
    \setlength\parindent{0mm}
    \setlength\parskip{5mm}

    %%% nadpis
    {\large\bfseries Abstrakt}

    %%% text abstrakt
    \noindent
    Tato bakalářská práce se zabývá vývojem intuitivního, uživatelsky orientovaného systému prodeje vstupenek s rezervací míst z pohledu frontendu s důrazem na interaktivní mapu sedadel.

    Teoretická část se zabývá identifikací klíčových částí takového systému a popisuje jejich funkčnost demonstrovanou na příkladu již existujících řešení.

    V rámci praktické části byl navržen a implementován prototyp webové aplikace umožňující nákup vstupenek s rezervací místa.
    Práce poskytuje detailní vhled do celkového procesu jak návrhu uživatelského rozhraní, tak implementace jednotlivých částí aplikace, včetně popisu použitých technologií a zvolené architektury.
    V závěru jsou identifikovány zajímavé výzvy, které se vyskytly při vývoji a jejich následné řešení, zhodnocení výsledného řešení a návrh možných budoucích vylepšení.

    \textit{Klíčová slova: interaktivní mapa sedadel, rezervace míst, vstupenky, webové aplikace, uživatelské rozhraní, frontend, TypeScript, React, SVG}
    \vss}

%%% abstrakt anglicky
\nobreak\vbox to 0.49\vsize{
    \setlength\parindent{0mm}
    \setlength\parskip{5mm}

    %%% nadpis
    {\large\bfseries Abstract}

    %%% text anglicky
    \noindent
    This bachelor's thesis deals with the development of an intuitive, user-oriented ticketing system with seat reservation from a front-end perspective with an emphasis on an interactive seat map.

    The theoretical part deals with the identification of the key parts of such a system and describes their functionality demonstrated on the example of already existing solutions.

    As part of the practical part, a prototype of a web application was designed and implemented enabling the purchase of tickets with seat reservation.
    The thesis provides a detailed insight into the overall process of both the design of the user interface and the implementation of individual parts of the application, including a description of the technologies used and the chosen architecture.
    In the conclusion, interesting challenges that occurred during the development and their subsequent solution, an evaluation of the resulting solution and a proposal for possible future improvements are identified.

    \textit{Keywords: interactive seat map, seat reservation, tickets, web application, user interface, frontend, TypeScript, React, SVG}
    \vss}

