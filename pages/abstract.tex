%%%
%%%  ABSTRAKT
%%%
%%%  * soubor obsahující abstrakt bakalářské práce
%%%
%%%  ===========================================================================
\newpage
\pagestyle{plain}

%%% abstrakt česky
\vbox to 0.5\vsize{
    \setlength\parindent{0mm}
    \setlength\parskip{5mm}

    %%% nadpis
    {\large\bfseries TODO: Abstrakt}

    %%% TODO: text abstrakt až po dopsání práce (lol)
    \noindent
    Tato bakalářská práce se zaměřuje na vývoj frontendové části webové aplikace pro prodej vstupenek s rezervací míst. Cílem práce je vyvinout prototyp aplikace, která potenciálnímu zákazníkovi umožní zobrazit mapu míst, vybrat si preferované místo, přidat vstupenky do nákupního košíku a následně vytvořit objednávku.

    Teoretická část práce se zaměřuje na obecnou problematiku prodeje vstupenek a moderního řešení pomocí platforem a služeb poskytujících online prodej vstupenek s možností rezervací míst. Tato část dále analyzuje trh současných poskytovatelů a definuje nejhlavnější technické problémy, které mohou při vývoji takového systému vzniknout a to výhradně z pohledu frontendu.

    Praktická část práce definuje rozsah implementované aplikace, podrobně popisuje hlavní funkce, komponenty, datové modely i některé nezbytné backendové části. Kapitola o návrhu uživatelského rozhraní popisuje principy, vzory a osvědčené postupy návrhu uživatelského rozhraní v kontextu vyvíjené aplikace. Kapitola o vývoji frontendové části poté podrobně popisuje technologie, nástroje a knihovny použité při vývoji aplikace.


    \vss}

%%% abstrakt anglicky
\nobreak\vbox to 0.49\vsize{
    \setlength\parindent{0mm}
    \setlength\parskip{5mm}

    %%% nadpis
    {\large\bfseries TODO: Abstract}

    %%% TODO: text anglicky
    \noindent
    This bachelor thesis deals with the implementation of a ticketing solution using seat reservations. The aim of
    the theoretical part of the thesis is to elaborate this issue and identify the main technical challenges and requirements for the implementation
    of such a system, especially from the frontend perspective. These challenges, such as intuitive user interface, availability of information
    real-time sales information, administration and management of seating plans or final booking and payment of orders, will be discussed in this thesis
    will be described in this thesis and options for their solution will be presented. The practical part will focus on the implementation of the part of the web application dealing with
    rendering of the interactive seating plan and its interaction with the customer. In this part the user interface design will be presented
    together with an introduction of the selected technologies to be implemented. Emphasis will be placed on the optimized design of the seating plan, the specification of
    of its data format, communcation with the API and overall documentation of the components used in the resulting application. The aim of this work is to describe
    the issues and challenges in implementing a web application addressing seat ticketing and present part of a real solution for such a
    application especially from the perspective of an interactive seat selection plan.

    Keywords: interactive seating plan, seat reservations, tickets, web applications, JavaScript, React, SVG
    \vss}

