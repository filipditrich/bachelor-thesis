%%%
%%%  ABSTRAKT
%%%
%%%  * soubor obsahující abstrakt bakalářské práce
%%%
%%%  ===========================================================================
\newpage
\pagestyle{plain}

%%% abstrakt česky
\vbox to 0.5\vsize{
    \setlength\parindent{0mm}
    \setlength\parskip{5mm}

    %%% nadpis
    {\large\bfseries Abstrakt}

    %%% TODO: text česky
    \noindent
    Tato bakalářská práce se zabývá problematikou implementačního řešení prodeje vstupenek s využitím rezervace míst. Cílem
    práce je v teoretické části tuto problematiku rozvést a identifikovat nejhlavnější technické výzvy a požadavky na implementaci
    takového systému a to zejména z pohledu frontendu. Tyto vývzy, typu intuitivní uživatelské prostředí, dostupnost informací
    o prodeji v reálném čase, administrace a správa sedačkových plánů či finální rezervace a platba objednávky, budou v této práci
    popsány a budou uvedeny možnosti jejich řešení. Praktická část bude zaměřena na implementaci části webové aplikace zabývající se
    vykreslováním interaktivního plánu sedaček a jeho interakce se zákazníkem. V této části bude představen návrh uživatelského rozhraní
    spolu s představením vybraných technologií k implementaci. Důraz bude kladen na optimalizované řešení plánu sedaček, specifikaci
    jeho datového formátu, komumnikaci s API a celkové dokumentaci komponent použitých ve výsledné aplikaci. Cílem této práce je popsat
    problematiku a vývzy při implementaci webové aplikace řešící sedačkový prodej vstupenek a představit část reálného řešení takovéto
    aplikace zejména z pohledu interaktivního plánu na výběr sedaček.

    Klíčová slova: interaktivní plán sedaček, rezervace míst, vstupenky, webové aplikace, JavaScript, React, SVG

    \vss}

%%% abstrakt anglicky
\nobreak\vbox to 0.49\vsize{
    \setlength\parindent{0mm}
    \setlength\parskip{5mm}

    %%% nadpis
    {\large\bfseries Abstract}

    %%% TODO: text anglicky
    \noindent
    This bachelor thesis deals with the implementation of a ticketing solution using seat reservations. The aim of
    the theoretical part of the thesis is to elaborate this issue and identify the main technical challenges and requirements for the implementation
    of such a system, especially from the frontend perspective. These challenges, such as intuitive user interface, availability of information
    real-time sales information, administration and management of seating plans or final booking and payment of orders, will be discussed in this thesis
    will be described in this thesis and options for their solution will be presented. The practical part will focus on the implementation of the part of the web application dealing with
    rendering of the interactive seating plan and its interaction with the customer. In this part the user interface design will be presented
    together with an introduction of the selected technologies to be implemented. Emphasis will be placed on the optimized design of the seating plan, the specification of
    of its data format, communcation with the API and overall documentation of the components used in the resulting application. The aim of this work is to describe
    the issues and challenges in implementing a web application addressing seat ticketing and present part of a real solution for such a
    application especially from the perspective of an interactive seat selection plan.

    Keywords: interactive seating plan, seat reservations, tickets, web applications, JavaScript, React, SVG
    \vss}

