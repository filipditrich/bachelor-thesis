%%%%% Nastavení kódování vstupních souborů: UTF-8
%%%%% ---------------------------------------------------------------
\usepackage[utf8]{inputenc}

%%%%% Nastavení textu
%%%%% ---------------------------------------------------------------
\usepackage{setspace}
\onehalfspacing
\usepackage{indentfirst}
\setlength{\parindent}{0pt}
\setlength{\parskip}{0.75\baselineskip}

%%%%% Nastavení barevných boxů/rámečku
%%%%% ---------------------------------------------------------------
\usepackage{tcolorbox}              %%% umožňuje vkládat rámečky
% gray box preset
\newtcolorbox{gray-box}[1]{colback=gray!5!white,colframe=gray!50!black,title=#1}
% user story box preset
\newcommand{\userstory}[4]{
    \begin{gray-box}{Uživatelský příběh #1 – #2}
        \textit{``#3``}
    \end{gray-box}
    #4
}

%%%%% Nastavení barev a syntaxe kódu
%%%%% ---------------------------------------------------------------
% TODO: Define the colors
\usepackage{fvextra}
\usepackage{xcolor}
\definecolor{codekeyword}{rgb}{0.0,0.3,0.7}
\definecolor{codestring}{rgb}{0.4,0.4,0.8}
\definecolor{codecomment}{rgb}{0.2,0.2,0.2}
\definecolor{codejsdoc}{rgb}{0.4,0.4,0.4}
\definecolor{codelinenum}{rgb}{0.8,0.8,0.8}
\definecolor{lightyellow}{rgb}{255, 255, 224}

% nastavení minted balíčku
\usepackage[outputdir=dist]{minted}
\setminted{
    frame=none,
    breaklines=true,
    fontsize=\footnotesize,
    tabsize=2,
    linenos,
    numbersep=5pt,
    xleftmargin=0pt,
    baselinestretch=1.2,
    style=friendly,
%   TODO: highlihgting not working
    highlightcolor=\color{lightyellow},
    keywordstyle=\color{codekeyword},
    stringstyle=\color{codestring},
    commentstyle=\color{codecomment}\itshape,
    morecomment=[s][\color{codejsdoc}]{/**}{*/},
    numberstyle=\footnotesize\color{codelinenum}
}

% nastavení listings balíčku
\usepackage{listings}
\lstset{
    basicstyle=\ttfamily,
    columns=fullflexible,
    frame=single,
    breaklines=true,
    showstringspaces=false,
    keywordstyle=\bfseries,
    commentstyle=\itshape\color{gray},
    captionpos=t
}

%%%%% Další užitečné balíčky
%%%%% ----------------------------------------------------------------
\usepackage{amsmath}                %%% rozšíření pro sazbu matematiky
\usepackage{amsfonts}               %%% matematické fonty
\usepackage{amsthm}                 %%% sazba vět, definic apod.
\usepackage{bm}                     %%% tučné symboly (příkaz \bm)
\usepackage{graphicx}               %%% vkládání obrázků
\usepackage[labelfont=bf]{caption}  %%% popisky obrázků, tabulek apod.
\newcommand{\source}[1][Vlastní zpracování]{\caption*{\hfill\footnotesize{Zdroj: \textit{#1}}}}
\usepackage{psfrag}                 %%% dodatečná úprava popisků v postscriptových obrázcích
\usepackage{fancyvrb}               %%% vylepšené prostředí pro strojové písmo
\usepackage[numbers]{natbib}                 %%% zajištuje možnost odkazovat na reference stylem AUTOR (ROK), resp. AUTOR [ČÍSLO]
\usepackage{usebib}                 %%% umožňuje používat bibliografické databáze
\usepackage{tikz}                   %%% vkládání vektorových obrázků
\usepackage{bbding}                 %%% balíček s nejrůznějšími symboly
\usepackage{icomma}                 %%% inteligetní čárka v matematickém módu
\usepackage{dcolumn}                %%% lepší zarovnání sloupců v tabulkách
\usepackage{booktabs}               %%% lepší vodorovné linky v tabulkách
\usepackage{paralist}               %%% lepší enumerate a itemize
\usepackage{float}                  %%% lepší práce s float objekty (obrázky, tabulky, ...)
\usepackage{subcaption}             %%% umožňuje vkládat podnadpisy k obrázkům a tabulkám
\usepackage{epigraph}               %%% umožňuje vkládat citáty
\newcommand\foreign[1]{\emph{#1}}   %%% zvýraznění cizích slov
\usepackage{pdfpages}               %%% umožňuje vkládat PDF soubory do dokumentu
\usepackage{nameref}                %%% umožňuje používat jmenné odkazy na kapitoly, sekce, ...
% vytvoří odkaz na kapitolu, sekci, ... včetně čísla a tučně
\renewcommand{\fullref}[1]{\textbf{\nameref{#1}}}

%%%%% Nastavení zkratek
%%%%% ---------------------------------------------------------------
\usepackage{acro}                   %%% balíček pro práci s akronymy
\DeclareAcronym{ui}{short=UI,       long=User Interface }
\DeclareAcronym{ux}{short=UX,       long=User Experience }
\DeclareAcronym{api}{short=API,     long=Application Programming Interface }
\DeclareAcronym{svg}{short=SVG,     long=Scalable Vector Graphics }
\DeclareAcronym{css}{short=CSS,     long=Cascading Style Sheets }
\DeclareAcronym{html}{short=HTML,   long=Hypertext Markup Language }
\DeclareAcronym{js}{short=JS,       long=JavaScript }
\DeclareAcronym{ts}{short=TS,       long=TypeScript }
\DeclareAcronym{json}{short=JSON,   long=JavaScript Object Notation }
\DeclareAcronym{spa}{short=SPA,     long=Single Page Application }
\DeclareAcronym{dom}{short=DOM,     long=Document Object Model }
\DeclareAcronym{mpa}{short=MPA,     long=Multi Page Application }
\DeclareAcronym{ssr}{short=SSR,     long=Server Side Rendering }
\DeclareAcronym{ssg}{short=SSG,     long=Static Site Generation }
\DeclareAcronym{seo}{short=SEO,     long=Search Engine Optimization }
\DeclareAcronym{http}{short=HTTP,   long=Hypertext Transfer Protocol }
\DeclareAcronym{url}{short=URL,     long=Uniform Resource Locator }
\DeclareAcronym{xsrf}{short=XSRF,   long=Cross-Site Request Forgery }
\DeclareAcronym{cd}{short=CD,       long=Continuous Development }
\DeclareAcronym{ws}{short=WS,       long=Websockets }

%%%%% Nastavení nadpisů
%%%%% ------------------------------------------------------------
\usepackage{titlesec}
%\titlespacing*{\chapter}{0pt}{-10mm}{5mm}
\titleformat{\chapter}{\normalfont\huge\bfseries}{\thechapter}{1em}{}
